\chapter{Quadratische Gleichungen}
	
	Autor: Marc Mittner
	
\noindent	\"Uberarbeitung: Marko Rak, Julia Hempel, Johannes Jendersie
	
	\section{Definition}
		
		Eine quadratische Gleichung ist eine Gleichung, die sich auf die Form 
		\[ ax^2 +bx+c=0 \]
		überführen lässt, mit  $ a, b, c \in \mathbb{R} $.
		
		Eine quadratische Gleichung ist in Normalform, falls $a = 1 $, also
        \begin{align*}
          x^2 &+ px + q=0 \quad\text{mit}\\
          p &= \frac b a \quad\text{und}\\
          q &= \frac c a
        \end{align*}
      
		
	\section{Lösen quadratischer Gleichungen}
		
		Jede quadratische Gleichung hat entweder keine, eine oder zwei reelle Lösungen.
	
		\subsection{Satz von Vieta:}
		Die reellen Zahlen $ a $ und $ b $ sind genau dann Lösungen der Gleichung 
		$x^2 + px + q = 0$, wenn für die Koeffizienten $ p $ und $ q $ gilt:
        \begin{align*}
            p &= -( a + b) \\
            q &= a \cdot b
        \end{align*}
		Daraus folgt:
		Hat eine quadratische Gleichung die Lösungen a und b, so lässt sie sich folgendermaßen darstellen:
		\[ (x - a)(x - b) = 0 \]
		Umgekehrt können die Lösungen aus dieser faktorisierten Form direkt abgelesen werden.
	
		\subsection{Mitternachtsformel}
			
			Jede quadratische Gleichung ($ ax^2 + bx + c = 0 $) kann mit Hilfe der Mitternachtsformel gelöst werden:
			
			\[x_{1,2} = \frac{-b \pm \sqrt{b^2 - 4ac}} {2a}\]
			
		\subsection{p-q-Formel}
			
			Jede quadratische Gleichung in Normalform ($ x^2 + px + q = 0 $) kann mit Hilfe der hergeleiteten p-q-Formel gelöst werden. Die Herleitung erfolgt mit der quadratischen Ergänzung:
			\[ x^2 + px + q = 0\]\\
			\[ x^2 + px + \left({}\frac{p}{2}\right)^2 - \left({}\frac{p}{2}\right)^2 + q = 0\]\\
			\[ \left({}x + \frac{p}{2} \right)^2 = \left({}\frac{p}{2}\right)^2 - q\]\\			
			\[ x_{1,2} = - \frac p 2 \pm \sqrt{\left(\frac p 2 \right)^2 - q} \]\\
			Zusammenhang mit der Mitternachtsformel:
            \begin{align*}
              p &= \frac b a\\
              q &= \frac c a
            \end{align*}

		\subsection{Satz vom Nullprodukt}
			
			Ein Produkt ist genau dann gleich Null, wenn einer seiner Faktoren gleich Null ist. 
			Lässt sich eine Gleichung auf die Form $(ax^2 + bx + c) \cdot x^k = 0$ bringen, so hat die Gleichung nach dem Satz vom Nullprodukt die Lösungen 			$x_{1,2,..,k} = 0$ und die Lösungen $x_{k+1}$ und $x_{k+2}$ können mit Hilfe der Mitternachtsformel / p-q-Formel gelöst werden. 
			
		\subsection{Substitution}
		
		Hat eine Gleichung die Form $ax^{2k} + bx^k + c = 0$, so kann $x^k$ durch eine Variable $u$ substituiert werden: 
		\[au^2 + bu + c = 0 \]
		Diese Gleichung kann dann als quadratische Gleichung gelöst werden. Für die Ergebnisse $ u_1 $ und $ u_2 $ gilt dann:
  \begin{align*}      
u_1 & = x^k& u_2 &= x^k \\
x_{1,2} &= \sqrt[k] {u_1} & x_{3,4} &= \sqrt[k] {u_2}
  \end{align*}

		Dabei gilt für die Anzahl der Lösungen:
		\begin{itemize}
			\item keine Lösung, wenn $ u < 0 $ und $k$ gerade
			\item eine Lösung, wenn $ -\infty < u < \infty $ und $k$ ungerade oder $ u = 0 $ und $k$ gerade.
			\item zwei Lösungen, wenn $ u > 0 $ und $k$ gerade
		\end{itemize}
		
	\section{Beispiele}
\begin{enumerate}
    \item $3x^2  + 3x - 36 = 0$
        
        Ausklammern: \newline
        $\begin{array}{crcl}
            &3(x^2 + x - 12) &=& 0 \\
            &x^2 + x - 12 &=& 0 \\
        \end{array}$ \newline\newline
        Lösen mit p-q-Formel ( $ p = 1 $ , $ q = -12 $ ):\newline
        $\begin{array}{crcl}
         &x_{1,2} &=& - \frac 1 2 \pm \sqrt{\left(\frac 1 2 \right)^2 + 12} \\
        &&=& - \frac 1 2 \pm \sqrt {\frac {49} 4}  \\
        &&=& - \frac 1 2 \pm \frac 7 2 \\
        &x_1 &=& 3 \\
        &x_2 &=& -4 \\
        \end{array}$ \newline\newline
        faktorisierte Darstellung: \newline
        $\begin{array}{crcl}
         &3 (x - 3) (x + 4) &=& 0\\
        \end{array}$
\end{enumerate}

\begin{itemize}        
    \item[2.] $x^7 + 19x^4 - 216x = 0 $\newline\newline
        Ausklammern:\newline
		$\begin{array}{crcl}
			&x(x^6+19x^3-216)&=&0\\
			&x_1&=&0
		\end{array}$           \newline  \newline   
        Substitution von \textit{$ x^3 = u $}: \newline
        $\begin{array}{crcl}
         &u^2 + 19u - 216 &=& 0 \\
        \end{array}$                \newline \newline
        Lösen mit p-q-Formel ($ p = 19 $ , $ q = -216 $): \newline 
        $\begin{array}{crcl}
         &u_{1,2} &=& - \frac {19} 2 \pm \sqrt{\left(\frac {19} 2 \right)^2 + 216} \\
         &&=& - \frac {19} 2 \pm \sqrt{\frac {361} 4 + \frac {864} 4}\\
         &&=& - \frac {19} 2 \pm \sqrt{\frac {1225} 4} \\
         &&=& - \frac {19} 2 \pm \frac {35} 2 \\
        &u_1 &=& 8 \\
        &u_2 &=& -27\\
        \end{array}$   \newline   \newline
        Resubstitution: \newline
        $\begin{array}{ll}
        &x^3 = u_1 \text{\ ergibt die Lösungen}\\
        &x^3 = 8 \\
        &x_2 = 2 \\[2ex]
        &x^3 = u_2 \text{\ ergibt die Lösungen}\\
        &x^3 = -27 \\
        &x_3 = -3\\
        \end{array}$
    
\end{itemize}

		
\section{Aufgaben}
\newcommand{\lineup}{\vspace{-1em}}
	
	%Autor: Marc Mittner
	%\"Uberarbeitung: Marko Rak
	
	Für alle Aufgaben gilt grundsätzlich:
	$ x, y, z \in \mathbb{R} $ sind Variable
	$ a, b, c \in \mathbb{R} $ sind feste Parameter
	
	Lösen Sie die folgenden Gleichungen:
	\begin{enumerate}\abovedisplayskip-1em
		\item 
		\[ x^2 - x - 2 = 0 \]
		\item
		\[4x^2 + 16x - 84 = 0 \]
		\item
		\[\frac 1 2 x^2 + 3x + 4 = 0 \]
		\item
		\[4x^2 + 48x + 144 = 0 \]
		\item 
		\[(x - \sqrt{157})^2 = 0 \]
		\item
		\[\frac 7 3 x^3 + \frac {49} 3 x^2 + 35x + 21 = 0 \]
		\item
		\[\frac 7 4 x^2 + 7x = -7 \]
		\item
		\[ |x^2| = 4 \]
		\item
		\[ |x|^2 = 4 \]
		\item
		\[ |x^2 - 4 | = 2 \]
		\item
		\[x^2 = x + 12 \]
		\item
		\[3x^2 + 4x + 1 = 0\]
		\item
		\[x^5 - 25x^3 + 144x = 0 \]
		\item
		\[(x - \pi)(x + \pi) = 0 \]
		\item
		\[\frac {x^3 - 2x^2} {x - 2} + \frac {2x^2 + 4x} {x + 2} = \-1\]
		\item
		\[x^4 - 14x^3 + 59x^2 - 70x = 0\]
		\item
		\[3x^7 - 42x^5 + 147x^3 = 0\]
		\item
		\[x^{12} = 4096\]
		\item
		\[x^4 + 4x^3 + 6x^2 + 4x + 1 = 0\]
		\item
		\[(\sqrt 2 x + 2 \sqrt 2)^2 = 0\]
		\item
		\[2ax^2 - 12ax + 18a = 0\]
		\item
		\[\frac 1 {x^2} + 1 = 2\]
		\item
		\[\frac 4 x + x = 4\]
	\end{enumerate}