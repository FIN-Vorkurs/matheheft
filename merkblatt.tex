\section{Komplexe Zahlen}

komplexe Zahlen:
\;$\displaystyle
\complex
\ =\ \{\;x+\imag y\;|\;x,y\in\real\;\}
\,.
$

kartesische Darstellung: $z=x+\imag y$,
\\
Realteil $\re(z)=x$, Imaginärteil $\im(z)=y$

Eulersche Darstellung: $z=r\e^{\imag\varphi}$,
\\
Betrag $\abs{z}=r\ge0$, Hauptargument $\arg{z}=\varphi\in\ocinterval{-\pi}{\pi}$

Zusammenhang: $r\e^{\imag\varphi}=r\,(\cos\varphi+\imag\sin\varphi)$, wobei $r=\sqrt{x^2+y^2}$ und $\varphi=\arctan(y/x)$ aus richtigem Quadranten:
\[
\begin{array}{c|c|c}
x & y & \varphi=\\
\hline
\ge0 & =0 & 0\\
<0 & =0 & \pi\\
=0 & >0 & \pi/2\\
=0 & <0 & -\pi/2
\end{array}
\qquad
\begin{array}{c|c|c}
x & y & \varphi\in\\
\hline
>0 & >0 & \ointerval{0}{\pi/2}\\
>0 & <0 & \ointerval{-\pi/2}{0}\\
<0 & >0 & \ointerval{\pi/2}{\pi}\\
<0 & <0 & \ointerval{-\pi}{-\pi/2}
\end{array}
\]
und umgekehrt: $x=r\cos\varphi$ und $y=r\sin\varphi$

konjugiert komplexe Zahl: $z=x-\imag y$

Addition: $(a+\imag b)+(c+\imag d)=(a+c)+\imag(b+d)$

Subtraktion: $(a+\imag b)-(c+\imag d)=(a-c)+\imag(b-d)$

Multiplikation: $(a+\imag b)\cdot(c+\imag d)=(ac-bd)+\imag(bc+ad)$
\\
bzw.\ $r\e^{\imag\varphi}\cdot{}s\e^{\imag\psi}=(rs)\,\e^{\imag(\varphi+\psi)}$

Division (erweitern mit konjugiert Komplexem des Nenners):
\\
$\displaystyle\frac{a+\imag b}{c+\imag d}
=\frac{ac+bd}{c^2+d^2}+\imag\,\frac{bc-ad}{c^2+d^2}$
\\
bzw.\ $\displaystyle\frac{r\e^{\imag\varphi}}{s\e^{\imag\psi}}
=\frac{r}{s}\,\e^{\imag(\varphi-\psi)}$ für $s>0$

Potenzen von $\imag$: Für $n\in\integer$ ist
\[
\imag^n
\ =\ \left\{\begin{array}{r@{\,,\; \text{falls}\quad}l}
1      & n\equiv0\!\mod4\\
\imag  & n\equiv1\!\mod4\\
-1     & n\equiv2\!\mod4\\
-\imag & n\equiv3\!\mod4
\end{array}\right.
\]
Potenzieren mit Exponenten $a\in\real$:
$
z^a=r^a\,\e^{\imag(a\varphi+2ak\pi)}
$
\\
und falls $a\in\integer$ speziell
$
z^a=r^a\,\e^{\imag a\varphi}
$

Radizieren: $n$-ten Wurzeln für $n=2,3,\dotsc$,
\[
\sqrt[n]{r}\,\left(
\cos\frac{\varphi+2k\pi}n
+\imag\sin\frac{\varphi+2k\pi}n
\right)
,\quad{}
k=0,\dotsc,n-1
\,,
\]
teilen den Kreis um Nullpunkt mit Radius $\sqrt[n]{r}$ in $n$ gleiche Teile

%================================================================================
\section{Kurvendiskussion}

Definitionsbereich: $D(f)$, Wertebereich: $W(f)=\{f(x)\,|\,x\in{}D(f)\}$

Definitionslücken? Stetigkeit? Differenzierbarkeit?

Nullstellen: $f(x_0)=0$

Extrema: $f'(x_{\e})=0$ und dann
\\
lokales Minimum: $f''(x_{\e})>0$, lokales Maximum: $f''(x_{\e})<0$
\\
globale Extrema: betrachte zusätzlich $\lim\limits_{x\to-\infty}f(x)$ und
$\lim\limits_{x\to\infty}f(x)$

Monotonie: wachsend: $f'(x)>0$, fallend: $f'(x)<0$

Krümmung: konvex: $f''(x)>0$, konkav: $f''(x)<0$

Wendestelle: $f''(x_{\w})=0$ und $f'''(x_{\w})\neq0$

weiteres: Polstellen? Asymptoten? Periodizität?

Skizzieren des Graphen!

