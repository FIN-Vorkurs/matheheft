
%=============================================================================
\chapter{Programmierung}\label{chap:prog}

\paragraph*{Autor:} Martin Knoll, Andreas Pfohl

%================================================================================
\section{Linux-Shellbefehle}
{
	\begin{tabular}{ll}
	\textbf{Befehl} & \textbf{Bedeutung}\\
	\hline
	man <Befehl> & manual (zeigt die Hilfedatei des Befehls an)\\
	cd <Pfad> & change directory (in ein Verzeichnis wechseln)\\
	ls <Pfad> & list (listen den Verzeichnisinhalt auf)\\
	rm <Datei> & remove (löscht Datei)\\
	mv & move (verschiebt Datei oder Verzeichnis)\\
	mkdir & make directory (legt ein neues Verzeichnis an)\\
	cat <Datei> & (gibt den Inhalt der Datei aus)\\
	touch <Dateiname> & (legt eine leere Datei an)\\
	file <Datei> & (gibt Informationen über Datei aus)\\
	passwd & password (ändern des Passworts)
	\end{tabular}
}

\section{Java}
	\subsection{Eclipse}
		\subsubsection{Projekt erstellen}
		\begin{enumerate}
		\item File \frqq New \frqq Java Project
		\item Projektnameame eintragen \frqq Finish
		\end{enumerate}
		
		\subsubsection{Klasse erstellen}
		\begin{enumerate}
		\item Rechtsklick auf das Projekt \frqq New \frqq Class
		\item Name und evtl. Package eigeben \frqq Finish
		\end{enumerate}

\section{Literatur}

\renewcommand{\labelenumi}{[\arabic{enumi}]}
\renewcommand{\theenumi}{[\arabic{enumi}]}
\begin{enumerate}
\item
I. N. Bronstein et al. \textit{Teubner-Taschenbuch der Mathematik}. (2 Bände)
B.\,G. Teubner Leipzig, 1996.

\item
K. Vetters. \textit{Formeln und Fakten im Grundkurs Mathematik}. B.\,G. Teubner Stuttgart, 2004.

\end{enumerate}
\pagestyle{scrplain}
\ofoot[]{}

\cleardoublepage
%\end{document}
