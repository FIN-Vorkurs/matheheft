
%=============================================================================
\chapter{Programmierung}\label{chap:prog}

\paragraph*{Autor:} Martin Knoll, Andreas Pfohl

%================================================================================
\section{Linux-Shellbefehle}
{
	\begin{tabular}{ll}
	\textbf{Befehl} & \textbf{Bedeutung}\\
	\hline
	man \textless{}Befehl\textgreater{} & manual (zeigt die Hilfedatei des Befehls an)\\
	cd \textless{}Pfad\textgreater{} & change directory (in ein Verzeichnis wechseln)\\
	ls \textless{}Pfad\textgreater{} & list (listen den Verzeichnisinhalt auf)\\
	rm \textless{}Datei\textgreater{} & remove (löscht Datei)\\
	mv & move (verschiebt Datei oder Verzeichnis)\\
	mkdir & make directory (legt ein neues Verzeichnis an)\\
	cat \textless{}Datei\textgreater{} & (gibt den Inhalt der Datei aus)\\
	touch \textless{}Dateiname\textgreater{} & (legt eine leere Datei an)\\
	file \textless{}Datei\textgreater{} & (gibt Informationen über Datei aus)\\
	passwd & password (ändern des Passworts)
	\end{tabular}
}

\section{Java}
	\lstset{language=Java}
	\subsection{Eclipse}
		\subsubsection{Projekt erstellen}
		\begin{enumerate}
		\item File \frqq New \frqq Java Project
		\item Projektnameame eintragen \frqq Finish
		\end{enumerate}
		
		\subsubsection{Klasse erstellen}
		\begin{enumerate}
		\item Rechtsklick auf das Projekt \frqq New \frqq Class
		\item Name und evtl. Package eigeben \frqq Finish
		\end{enumerate}

		\subsubsection{Programm testen}
		\begin{enumerate}
		\item Normales Ausführen: Grüner Pfeil
		\item Debugmodus: Käfer
		\end{enumerate}

	\subsection{Java Referenz}
		\subsubsection{Operatoren}
		\begin{tabular}{ll}
		\textbf{Operator} & \textbf{Wirkung}\\
		\hline
		\lstinline$=$ & Zuweisung, weist den Wert der rechten Seite der linken Seite zu.\\
		\lstinline$+, +=, ++$ & Addition, Addition mit Zuweisung, Inkrementierung\\
		\lstinline$-, -=, --$ & Subtraktion, Subtraktion mit Zuweisung, Dekrement\\
		\lstinline$*, *=$ & Multiplikation, Multiplikation mit Zuweisung\\
		\lstinline$/, /=$ & Division, Division mit Zuweisung\\
		\lstinline$return$ & Verlasse die Methode (und gib Wert zurück)\\
		\end{tabular}

		\subsubsection{Schleifen}
		\begin{tabular}{ll}
		\textbf{Art} & \textbf{Java Syntax}\\
		\hline
		Kopfgesteuerte Schleife & \lstinline$while (Bedingung) {Befehle}$\\
		Fußgesteuerte Schleife & \lstinline$do {Befehle} while (Bedingung)$\\
		Zählschleife & \lstinline$for (int i = 0; i <= 5; i++) {Befehle}$\\
		Iteratorschleife & \lstinline$for (int i : arr) {Befehle}$\\
		
		\end{tabular}

		\subsubsection{API Methoden}
		\begin{tabular}{ll}
		\textbf{Methode} & \textbf{Wirkung}\\
		\hline
		\lstinline$System.out.println(String)$ & Gibt String in der Konsole aus\\
		
		\end{tabular}

		\subsubsection{Achtung!}
			Diese Übersicht umfasst nur die wichtigsten Befehle, und ist bei weitem nicht vollständig. Eine Ausführlichere Dokumentation der Java API findet ihr unter http://download.oracle.com/javase/7/docs/api/

	\subsection{Übungsaufgaben}
		\subsubsection{Potenzieren}
			Schreibe eine Methode \lstinline$long pot(long b, int p)$, welche einen gegeben Wert b mit einem beliebigen Wert p potenziert. Schreibe außerdem eine Methode \lstinline$long square(long b)$ zum quadrieren und eine Funktionion \lstinline$long cubic(long b)$.
		\subsubsection{Gröster gemeinsamer Teiler}
			Schreibe die Methode \lstinline$long ggt(long a, long b)$ zur Berechnung des größten gemeinsamen Teilers auf Basis seiner rekursiven Definition.


\section{Literatur}

\renewcommand{\labelenumi}{[\arabic{enumi}]}
\renewcommand{\theenumi}{[\arabic{enumi}]}
\begin{enumerate}
\item
I. N. Bronstein et al. \textit{Teubner-Taschenbuch der Mathematik}. (2 Bände)
B.\,G. Teubner Leipzig, 1996.

\item
K. Vetters. \textit{Formeln und Fakten im Grundkurs Mathematik}. B.\,G. Teubner Stuttgart, 2004.

\end{enumerate}
\pagestyle{scrplain}
\ofoot[]{}

\cleardoublepage
%\end{document}
