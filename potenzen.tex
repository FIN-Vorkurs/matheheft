\section{Potenzen}
\label{Potenzen}
Autor: Katja Matthes
%%%%%%%%%%%%%%%%%%%%%%%%%%%%%%%%%%%%%%%%%%%%%%%%%%%%%%%%%%%%%%%%%%%%
%%%Definition
%%%%%%%%%%%%%%%%%%%%%%%%%%%%%%%%%%%%%%%%%%%%%%%%%%%%%%%%%%%%%%%%%%
\subsection{Definition}
Potenzen sind eine abkürzende Schreibweise für eine wiederholte Multiplikation mit einem Faktor.
   \[ \begin{matrix} \underbrace{ a\cdot a\cdot a\cdots a }&=a^n\\{n\ \mathrm{Faktoren}} \end{matrix} \]
\begin{multicols}{3}
	$ a^n $ \ldots\ Potenz \\
  $ a $ \ldots\ Basis\\
  $ n $ \ldots\ Exponent
\end{multicols}
%%%%%%%%%%%%%%%%%%%%%%%%%%%%%%%%%%%%%%%%%%%%%%%%%%%%%%%%%%%%%%%%%%%%
%%%Besondere Exponenten
%%%%%%%%%%%%%%%%%%%%%%%%%%%%%%%%%%%%%%%%%%%%%%%%%%%%%%%%%%%%%%%%%%
\subsection{Besondere Exponenten}
Seien $ a \in \mathbb{R}\setminus\{0\} $ und $ n \in \mathbb{N}_{0} $, dann gilt:
\begin{align*}
    a^0 &= 1 \\
    a^1 &= a \\
    a^{-n} &= \frac{1}{a^n} 
\end{align*}
%%%%%%%%%%%%%%%%%%%%%%%%%%%%%%%%%%%%%%%%%%%%%%%%%%%%%%%%%%%%%%%%%%%%
%%%Potenzgesetze
%%%%%%%%%%%%%%%%%%%%%%%%%%%%%%%%%%%%%%%%%%%%%%%%%%%%%%%%%%%%%%%%%%
\subsection{Potenzgesetze}
Folgende \textbf{Potenzgesetze} gelten für alle $ m $, $ n \in \mathbb{Z} $ und $ a$, $ b \in \mathbb{R}\setminus\{0\} $.
%%%%%%%%%%%%%%%%%%%%%%%%%%%%%%%%%%%%%%%%%%%
\begin{enumerate}
	\item %%%%%%%%%%% 1. %%%%%%%%%%%%%%%%%
		Potenzen mit gleicher Basis werden multipliziert, indem die Basis beibehalten wird und die Exponenten addiert werden.
		\[a^m \cdot a^n = a^{m+n}\]
	\item %%%%%%%%%%% 2. %%%%%%%%%%%%%%%%%
		Potenzen mit gleicher Basis werden dividiert, indem die Basis beibehalten wird und die Exponenten subtrahiert werden.	
		\[\frac{a^m}{a^n} = a^{m-n}\]
	\item %%%%%%%%%%% 3. %%%%%%%%%%%%%%%%%
		Potenzen mit gleichem Exponenten werden multipliziert, indem die Basen multipliziert werden und der Exponent beibehalten wird.
		\[a^n \cdot b^n = (a \cdot b)^n\]
	\item %%%%%%%%%%% 4. %%%%%%%%%%%%%%%%%
		Potenzen mit gleichem Exponenten werden dividiert, indem die Basen dividiert werden und der Exponent beibehalten wird.
		\[\frac{a^n}{b^n} = \left(\frac{a}{b}\right)^n\]
	\item %%%%%%%%%%% 5. %%%%%%%%%%%%%%%%%
		Potenzen werden potenziert, indem die Basis beibehalten wird und die Exponenten multipliziert werden.
		\[(a^m)^n = a^{m \cdot n} = (a^n)^m \]
\end{enumerate}
%%%%%%%%%%%%%%%%%%%%%%%%%%%%%%%%%%%%%%%%%%%%%%%%%%%%%%%%%%%%%%%%%%%%
%%%Wurzeln
%%%%%%%%%%%%%%%%%%%%%%%%%%%%%%%%%%%%%%%%%%%%%%%%%%%%%%%%%%%%%%%%%%
\subsection{Wurzeln}
Seien $ m $, $ n \in \mathbb{N} $ und $ a \in \mathbb{R} $ mit $ a>0 $, dann gilt:
	\[a^{\frac{m}{n}} = \sqrt[n]{a^m}\]
Damit sind die Potenzgesetze auch auf Wurzeln anzuwenden.
\noindent Man nennt $a$ den Radikanten und $n$ den Wurzelexponenten.
%%%%%%%%% Wurzelgesetzte %%%%%%%%%%%%%%%%%%%%%%%%%%%%%
\subsection{Wurzelgesetze}
Für $m$, $n \in \mathbb{N}_{>1}$ und nichtnegativen reelen Radikanden $a$ und $b$ gilt:
\begin{enumerate}
	\item $\sqrt[m]{a}\cdot\sqrt[n]{a} = \sqrt[mn]{a^{m+n}}$
	\item $\frac{\sqrt[m]{a}}{\sqrt[n]{a}} = \sqrt[mn]{a^{n-m}}$
	\item $\sqrt[n]{a}\cdot\sqrt[n]{b} = \sqrt[n]{a\cdot b}$
	\item $\frac{\sqrt[n]{a}}{\sqrt[n]{b}} = \sqrt[n]{\frac{a}{b}}$
	\item $\sqrt[n]{\sqrt[m]{a}} = \sqrt[mn]{a} = \sqrt[m]{\sqrt[n]{a}}$
\end{enumerate} 
%%%%%%%%%%%%%%%%%%%%%%%%%%%%%%%%%%%%%%%%%%%%%%%%%%%%%%%%%%%%%%%%%%%%
%%%Literatur
%%%%%%%%%%%%%%%%%%%%%%%%%%%%%%%%%%%%%%%%%%%%%%%%%%%%%%%%%%%%%%%%%%
\subsection{Literatur}
\begin{itemize}
	\item http://de.wikipedia.org/wiki/Potenzen
	\item Formeln und Tabellen für die Sekudarstufen I und II. 7. Aufl. - Berlin: Paetec, Ges. für Bildung und Technik, 1999
\end{itemize}
%%%%%%%%%%%%%%%%%%%%%%%%%%%%%%%%%%%%%%%%%%%%%%%%%%%%%%%%%%%%%%%%%%%%
%%%Aufgaben
%%%%%%%%%%%%%%%%%%%%%%%%%%%%%%%%%%%%%%%%%%%%%%%%%%%%%%%%%%%%%%%%%%
\pagebreak
\subsection{Aufgaben}

\subsubsection{Aufgabe 1}
Vereinfache.
\begin{enumerate}
\begin{multicols}{2}
	\item \quad $ 3x^4 - x^4 - x^3(x + 2) $
	\item \quad $ -12a^2 + 3a(a + 1) $
	\item \quad $ ax^n + 4x^n $
	\item \quad $ (1-t)^2 - \frac{1}{2}(1-t)^2 $
	\item \quad $ a(x+t)^k - b(x+t)^k $
	\item \quad $ tx^3 - 3x^2 + 2tx^3 - 4x^2 $
	\item \quad $ t^3 \cdot t^4 - t^5(t^2+1) $
	\item \quad $ x^2 \cdot x^3 \cdot x^4 $
	\item \quad $ 3a^k \cdot a^{k-1} \cdot a $
	\item \quad $ b^n \cdot b^{2n+1} $
	\item \quad $ (x+1)^{n-1} \cdot (x+1)^{n+1} $
	\item \quad $ \left(\frac{x}{3}\right)^4 \cdot \left(\frac{x}{3}\right)^2 $
	\item \quad $ t^2 \cdot x^2 \cdot t^n \cdot x^{n-1} $
	\item \quad $ a \cdot b^k \cdot a^{2n} \cdot b^{k-3} $
	\item \quad $ (x-2)^n \cdot (x-2)^{1-n} $
	\item \quad $ 0,3^6 \cdot \left(\frac{10}{3}\right)^6 $
	\item \quad $ 2^x \cdot \left(\frac{5}{2}\right)^x \cdot 5 $
	\item \quad $ 2^5 \cdot \left(\frac{1}{2}\right)^4 $
	\item \quad $ \left(\frac{x}{4}\right)^4 \cdot 4^6 $
	\item \quad $ 2^n \cdot \left(\frac{x}{2}\right)^n \cdot x $
	\item \quad $ 9 \cdot 3^{n+1} $
	\item \quad $ (a-b)^9 \cdot (a-b) $
	\item \quad $ \left(\frac{a-b}{c}\right)^{2k} \cdot \left(\frac{c}{a-b}\right)^{2k} $
\end{multicols}
\end{enumerate}

%\newpage

\subsubsection{Aufgabe 2}
Vereinfache.
\begin{enumerate}
\begin{multicols}{2}
	\item \quad $ \frac{a^6}{a^3} $
	\item \quad $ \frac{x^{2n+1}}{x^n} $
	\item \quad $ \frac{15e^{x+1}}{5e^x} $
	\item \quad $ \frac{x^4}{x^7} $
	\item \quad $ \frac{2a^{1-2n}}{4a^{n+1}} $
	\item \quad $ \frac{a^4b^{4n+3}}{a^nb^{2n-1}} $
	\item \quad $ \frac{81}{3^{x+3}} $
	\item \quad $ \frac{(a-b)^3}{(a-b)^{n-1}} $
	\item \quad $ \frac{(ab)^3}{x^2y} \cdot \frac{(xy)^2}{a^4b^2} $
	\item \quad $ \frac{a^{n+1}}{a^n} $
	\item \quad $ \frac{10^3}{2^3} $
	\item \quad $ \frac{2,5^4}{0,5^4} $
	\item \quad $ \frac{(10ab)^k}{(4b)^k} $
	\item \quad $ \left(\frac{a}{b}\right)^n \cdot \frac{a}{b} $
	\item \quad $ \left(\frac{-1}{a-b}\right)^3 $
	\item \quad $ \left(\frac{x}{2}\right)^3 : \left(\frac{x}{3}\right) $
	\item \quad $ (-5^2)^3 $
	\item \quad $ 3(c^4)^3 - 6c^{12} $
	\item \quad $ (3b^2c^{n-1})^4 $
	\item \quad $ \left(\frac{7a^2}{49b^3}\right)^2 $
	\item \quad $ \left(\frac{-1}{c^3}\right)^{2n} $
	\item \quad $ (3b^{n+1} \cdot c^{n-1})^2 $
	\item \quad $ (x^2y^3z^2)^5 $
	\item \quad $ (0,5e^{x+2})^2 $
	\item \quad $ \left(\frac{2}{x^2}\right)^5 - \left(\frac{3}{x^5}\right)^2 $
	\item \quad $ \left[\left(-\frac{3}{t}\right)^3\right]^4 \cdot \frac{t^9}{81} $
	\item \quad $ \frac{(ab)^2}{x^3y} \cdot \frac{x^5y^2}{a^2b} $
	\item \quad $ \frac{(4-12x)^3}{64} $
	\item \quad $ \frac{(2x-4)^5}{(2-x)^3} $
	\item \quad $ \frac{(4ab)^4}{(6a^2)^4} \cdot \frac{5}{b^4} $
	\item \quad $ (a-b^2) \cdot (a-b^2)^n $
\end{multicols}
\end{enumerate}

\subsubsection{Aufgabe 3}
Vereinfache.
\begin{enumerate}
	\item \quad $ \left(\frac{1}{2}x^2\right)^5 + \frac{1}{8}(x^2)^5 + (2x^5)^2 $
	\item \quad $ \frac{1}{4} \cdot 2^4(2^2)^3 $
	\item \quad $ (3^{n+1})^2 $
	\item \quad $ (3x^2 - 5x)(1-x^3) + (x^2 + 3x^4)x^3 $
	\item \quad $ a^{2r}b^r(a^{2r} - a^rb^{r+1} + b^{2r+2}) $
\end{enumerate}

\subsubsection{Aufgabe 4}
Vereinfache.
\begin{enumerate}
\begin{multicols}{2}
	\item \quad $ -3x^3 \cdot x^2 + 5x \cdot x^4 $
	\item \quad $ 4t^{n-4}t^3-t \cdot t^{n-2} $
	\item \quad $ 2x^5y^3y - 4x^3y^2x^2y^2 $
	\item \quad $ \frac{4x^5 + 6x^4 -12x^2}{2x^2} $
	\item \quad $ (9 \cdot 3^n - 3^{n+1}) : 3^{n-1} $
	\item \quad $ (2x+6)^2+(x+3)^2 $
	\item \quad $ \frac{5a-20}{4a-16} $
	\item \quad $ (3t^2 - 3t^3)^2 $
\end{multicols}
\end{enumerate}

\newpage

\subsubsection{Aufgabe 5}
Faktorisiere - Schreibe als Produkt durch Ausklammern.
\begin{enumerate}
\begin{multicols}{2}
	\item \quad $ 3a^2 + 6a^3 $
	\item \quad $ \frac{1}{2}e^x - \frac{1}{4}e^{x+1} $
	\item \quad $ a^{5b} + 3a^b $
	\item \quad $ 2^x + 2^{x+1} $
	\item \quad $ x^4+2x^3 $
	\item \quad $ x^{n+3} - 4x^{n+2} $
	\item \quad $ -6t^{n+2}+18t^{2-n} $
	\item \quad $ e^x-e^{3x} $
\end{multicols}
\end{enumerate}

\subsubsection{Aufgabe 6}
Vereinfache.
\begin{enumerate}
\begin{multicols}{2}
	\item \quad $ \frac{x^4-x^3}{x^2-x} $
	\item \quad $ \frac{e^{3x}+e^{2x}}{e^{2x}} $
	\item \quad $ \frac{a^7b^3-ab^7}{a^5b-a^2b^4} $
	\item \quad $ \frac{32}{2^{n+5}} + \frac{2^{-n+3}}{8} $
\end{multicols}
\end{enumerate}

\subsubsection{Aufgabe 7}
Berechne y.
\begin{enumerate}
	\item \quad $ y = \frac{1}{4}x^4-2tx^3+\frac{9}{2}t^2x^2 $ mit $ x = 3t $
	\item \quad $ y = e^{x^2-t^2}+3e^{5t-(t-x)} $ mit $ x = -t $
	\item \quad $ y = \frac{3}{2t^2}x^4 - \frac{4}{t}x^3 + 3x^2 - 4 $ mit $ x = \frac{1}{3}t $
	\item \quad $ y = \frac{e^{3tx}+4e^3}{tx-4} $ mit $ x = \frac{1}{t} $
	\item \quad $ y = \frac{tx^3}{2(x+t)^2} $ mit $x = -3t $ 
\end{enumerate}

\subsubsection{Aufgabe 8}
Klammere aus.
\begin{enumerate}
	\item \quad $ a^n+a^{4-n}+a^{2n} = a^{2n}(\ldots) $
	\item \quad $ a^3 + a^{1-n} + a^{n+4} = a^{n+3}(\ldots) $
	\item \quad $ \frac{3}{2}x^4+\frac{3}{4}x^3+\frac{1}{8}x^2 = \frac{1}{8}x^2(\ldots) $
	\item \quad $ e^{3x}-2e^{-x} = e^{-x}(\ldots) $
	\item \quad $ te^{2x}-2e^{x+1} = e^x(\ldots) $
\end{enumerate}

\subsubsection{Aufgabe 9}
Multipliziere aus und vereinfache.
\begin{enumerate}
\begin{multicols}{2}
	\item \quad $ \frac{1}{4} \cdot 2^{-4} \cdot (2^2)^3 $
	\item \quad $ (e^x - e^{-x} + 5)e^x $
	\item \quad $ 2^x(2^{-1}+2^x) $
	\item \quad $ (x^4+x^{-2})(x^3-x^{-3}) $
\end{multicols}
\end{enumerate}

\subsubsection{Aufgabe 10}
Vereinfache/Fasse zusammen.
\begin{enumerate}
\begin{multicols}{2}
	\item \quad $ a^2 \cdot (a^2)^{-2} + 3a \left(\frac{1}{a}\right)^3 $
	\item \quad $ \frac{1}{18} \cdot (3^2)^2 + \frac{1}{2} \cdot 3^3 \cdot \left(\frac{1}{3}\right)^2 $
	\item \quad $ (x^2 \cdot x^{-3})^{-2} + \left(\frac{3}{x^2}\right)^{-1} $
	\item \quad $ a^5 \cdot a^{-2} + 4a^2 \cdot a $
	\item \quad $ \left(\frac{2}{x}\right)^3 + \left(\frac{1}{x}\right)^3 $
	\item \quad $ \frac{1}{e^{2x}} + 3(e^{-x})^2 - \left(\frac{2}{e^x}\right)^2 $
	\item \quad $ e^{-x} \cdot e^{-x+2} \cdot e^{2x-3} $
	\item \quad $ 6x^3 \cdot x^{-1} - 8x^4 \cdot x^{-2} $
	\item \quad $ (t^7-t^4) \cdot t^{-3} $
\end{multicols}
\end{enumerate}
	
\subsubsection{Aufgabe 11}
Vereinfache/Fasse zusammen.
\begin{enumerate}
\begin{multicols}{2}
	\item \quad $ \frac{-2^3 - 2 \cdot 4}{2 \cdot 2^3} $
	\item \quad $ \frac{(1-x)^2}{(x-1)} $
	\item \quad $ \frac{e^{3x+1}}{e^{-x+2}} $
	\item \quad $ \frac{1,5e^{3x} - e^x}{1,5e^{3x}} $
\end{multicols}
\end{enumerate}

\subsubsection{Aufgabe 12}
Vereinfache/Fasse zusammen.
\begin{enumerate}
	\item \quad $ a^4 \cdot a^{-6} - 3a^3 \cdot a^{-5} + a^2 $
	\item \quad $ (a^{n+2} - 4a^n - 2a^{2-n})\cdot \frac{a^{-2}}{2} $
	\item \quad $ 4x^{-4}x^7 - 0,5 x^4x^{-1} + \left(\frac{1}{x^2}\right)^{1,5} $
	\item \quad $ \frac{a^{n+1}}{a} + \frac{a^{2n-1}}{a^{n+2}} + (a^{n-1})^2 \cdot a^{2-n} $
	\item \quad $ \frac{2^{2k}}{8} \cdot 2^{3-k} + 2 \cdot 2^{k-1} $
\end{enumerate}

\newpage

\subsubsection{Aufgabe 13*}
Vereinfache. (Tipp: Mache eine Fallunterscheidung.)
\begin{enumerate}
\begin{multicols}{2}
	\item \quad $ (a-b)^n + (b-a)^n $
	\item \quad $ (x-2)^n + (2x-4)^n - (2-x)^n $
\end{multicols}
\end{enumerate}