\chapter{Funktionen Loesungen}

\section{Trigonometrische Funktionen}
\begin{enumerate}
	\item
	% SEITE 71 AUFGABE 1
	
	
%		Ben\"otigt wird: 
%		\begin{align*}
%		\sin x &= \sin(\pi-x) \\
%		\cos x &= -\cos(\pi-x) \\
%		\cos(-x) &= \cos(x) \\
%		\sin(-x) &= -\sin(x) \\
%		\tan x  &= \frac{\sin x} {\cos x} \\
 % 	\end{align*} 
  	 \begin{enumerate}
  	 
  	 \item $\sin \frac{2\pi}{3} = \sin(\pi - \frac{2\pi}{3}) = \sin(\frac{\pi}{3}) = \frac{1}{2}\sqrt{3}$
		 \item $\sin \frac{5\pi}{6} = \sin(\pi - \frac{5\pi}{6}) = \sin(\frac{\pi}{6}) = \frac{1}{2}\sqrt{1}$
   	\item $\sin \pi = \sin (\pi - \pi) = 0$
   	\item $\sin \frac{3\pi}{2} = \sin(\pi - \frac{3\pi}{2}) = \sin(-\frac{\pi}{2}) = -\sin (\frac{\pi}{2})=- \frac{1}{2} \sqrt{4}$
   	\item $\sin \frac{11\pi}{6} =\sin(\pi - \frac{11\pi}{6}) = \sin(-\frac{5\pi}{6}) = -\sin (\pi - \frac{5\pi}{6})= -\sin (\frac{\pi}{6}) = -\frac{1}{2}\sqrt{1}$
   \item $\sin \frac{7\pi}{3} =\sin(\pi - \frac{7\pi}{3}) = \sin(-\frac{4\pi}{3}) = -\sin (\pi - \frac{4\pi}{3})= -\sin (-\frac{1\pi}{3}) = \sin (\frac{\pi}{3}) = \frac{1}{2}\sqrt{3}$
   \item $\sin \frac{29\pi}{6} =\sin(\pi - \frac{29\pi}{6}) = \sin(-\frac{23\pi}{6}) = -\sin (\pi - \frac{23\pi}{6})= -\sin (-\frac{17\pi}{6}) = \sin(\pi - \frac{17\pi}{6}) = \sin(-\frac{11\pi}{6}) = -\sin (\pi - \frac{11\pi}{6})= -\sin (-\frac{5\pi}{6}) = \sin (\pi - \frac{5\pi}{6})= \sin (\frac{\pi}{6}) = \frac{1}{2}\sqrt{1}$
   \item $\sin (-\frac{3\pi}{4}) = -\sin (\frac{3\pi}{4}) = -\sin(\pi - \frac{3\pi}{4}) = -\sin(\frac{\pi}{4}) = -\frac{1}{2}\sqrt{2}$
   \item $\cos \frac{\pi}{6} = \frac{1}{2}\sqrt{3}$
   \item $\cos \frac{\pi}{4} = \frac{1}{2}\sqrt{2}$
   \item $\cos \frac{\pi}{3} = \frac{1}{2}\sqrt{1}$
   \item $\cos \frac{\pi}{2} = \frac{1}{2}\sqrt{0}$
   \item $\cos \frac{11\pi}{6} = -\cos (\pi -\frac{11\pi}{6}) = - \cos (\frac{5\pi}{6}) = cos(\frac{1\pi}{6}) = \frac{1}{2}\sqrt{3}$
   \item $\cos \frac{3\pi}{4} = -\cos (\pi -\frac{3\pi}{4}) = - \cos (\frac{1\pi}{4}) = - \frac{1}{2}\sqrt{2}$
   \item $\cos \frac{2\pi}{3} = -\cos (\pi - \frac{2\pi}{3}) = - \cos (\frac{2\pi}{3}) = - \frac{1}{2}\sqrt{1}$
   \item $\cos \frac{4\pi}{6} = -\cos (\pi - \frac{2\pi}{3})= - \cos (\frac{2\pi}{3}) = - \frac{1}{2}\sqrt{1}$
   \item $\cos \frac{7\pi}{3} = -\cos (\pi - \frac{7\pi}{3}) = \cos (\pi - \frac{4\pi}{3}) = \cos(\frac{1\pi}{3}) =  \frac{1}{2}\sqrt{2}$
   \item $\cos -\frac{11\pi}{4} = \cos (\frac{11\pi}{4}) = - \cos(\pi - \frac{11\pi}{4}) = \cos (\pi - \frac{7\pi}{4}) = \cos -\frac{3\pi}{4}) = - \cos( -\frac{1\pi}{4}) = - \frac{1}{2}\sqrt{2}$
   \item $\tan \frac{\pi}{6} = \frac{\sin \frac{\pi}{6}}{\cos \frac{\pi}{6}} = \frac{\frac{1}{2} \sqrt{1}}{\frac{1}{2} \sqrt{3}}= \frac{1}{\sqrt{3}})$
   \item $\tan -\frac{\pi}{3}= \frac{\sin \frac{-\pi}{6}}{\cos \frac{-\pi}{6}}=\frac{-\sin \frac{\pi}{6}}{\cos \frac{\pi}{6}} = - \frac{\frac{1}{2}\sqrt{3}}{\frac{1}{2}\sqrt{1}} = - \sqrt{3}$
  \end{enumerate}
  
  



\item

% SEITE 71 AUFGABE 2
%\begin{minipage}{6cm}

\begin{tabular}{ccccc}
$\alpha$ & $\beta$ & $a$ & $b$ & $c$ \\
\hline

$\frac{\pi}{4}$ & $\frac{\pi}{4}$ & $1$ & $1$ & $\sqrt{2}$ \\
$\frac{\pi}{6}$ & $\frac{\pi}{3}$ & $2$ & $\sqrt{12}$ & $4$\\ 
kein Dreieck &  & $\frac{1}{2}\sqrt{3}$ &  & $\frac{1}{2}$\\
$53,13^\circ$ & $26,87^\circ$ & $4$ & $3$ & $5$ \\
$\frac{\pi}{6}$ & $\frac{\pi}{3}$ & $1$ & $\sqrt{3}$ & $2$ \\
$\frac{\pi}{6}$ & $\frac{\pi}{3}$ & $\frac{2}{\sqrt{3}}$ & $2$ & $\frac{4}{\sqrt{3}}$ \\

%\end{minipage}
\end{tabular}
 
 % SEITE 71 AUFGABE 3
\item
 \begin{align*}
 %\sin(2\alpha) & = 2 \sin \alpha \cos \alpha \\
 %\\
 \sin(4\alpha) & = 2 (\sin(2\alpha) \cos(2\alpha)) \\
 							 & = 4 \sin(\alpha) \cos(\alpha) \cos(2\alpha) \\
 							 & = 4 \sin (\alpha) \cos (\alpha) (\cos^2(\alpha) - \sin^2 (\alpha)) \\
 							 & = 4 \sin (\alpha) \cos^{3} (\alpha) - 4 \sin ^{3} (\alpha) \cos (\alpha)	\\
 	\sin(4\alpha) & = 4 (\sin (\alpha) \cos^{3} (\alpha) - \sin ^{3} (\alpha) \cos (\alpha))	\\
 \end{align*}
 
\item %SEITE 71 Aufgabe 4
 
 \par Ich l\"ose die Aufgabe mal von hinten - muss auch andersrum gehen...
 \begin{align*}
 2\cos^2\alpha -1 & = 2\cos^2\alpha - (\sin^2\alpha + \cos^2\alpha) \\
 									& = \cos^2\alpha - \sin^2\alpha \\
 							 & = \cos\alpha \cos\alpha  - \sin\alpha \sin\alpha \\
 							 & = \cos(\alpha+\alpha) \\
 							 & = \cos(2\alpha) \\
 \end{align*}
 
 
\item %SEITE 71 Aufgabe 5
 \par
 Wenn die Seitenl\"ange des W\"urfels maximal sein soll, muss der Durchmesser der Kugel gleich der Diagnoalen des W\"urfels sein.
 \newline
 \begin{align*}
 r &= 1  \text{   (Radius der Kugel)}\\
 a &= \text{   (Kante des Wuerfels)}\\
 D &= a\sqrt{3} \text{   (Diagonale des Wuerfels)}\\
 \\
 2r &= a\sqrt{3}\\
 a &= \frac{2}{\sqrt{3}}\\
 \end{align*} 
 \end{enumerate}
   
   
   
 \subsection{Exponentialfunktionen und Logarithmus}
  \begin{enumerate}
	\item %SEITE 74 AUFGABE 1
	\begin{enumerate}
   \item $1=e^x \ x=0$
   \item $8=2^x \ x=3$
   \item $3=5e^x \ x=ln(\frac{3}{5})$
   \item $e=\frac{e^x}{e} \ x=2$   
   \item $9=e^{cx} \ x=\frac{ln(9)}{c}$ 
   \item $3=log_2(x)  \ x=8$
   \item $0=log_{42}(x) \ x=1$
   \item $0=5log_5(x) \ x=1$
   \item $9=3ln(e^x) \ x=3$
   
 
 \end{enumerate} 
  \item %SEITE 74 AUFGABE 2
  \begin{enumerate}
  
  \item $lg2+lg5=lg10=1$
  \item $lg5+lg6-lg3=lg10=1$
  \item $3lna+5lnb-lnc=ln(\frac{a^3b^5}{c})$ f\"ur $c\neq0$
  \item $2lnv-lnv=ln(v^2)-lnv=ln(v)$  f\"ur $v\neq0$
  \item $\frac{1}{2}log_7 9 - \frac{1}{4} log_7 81=log_7 \sqrt9 - log_7\sqrt[4]{81}=0 $
  \item \begin{align*} 
  log_3(x-4)+log_3(x+4) &= 3\\
  log_3((x-4)(x+4)) &= 3\\
  log_3(x^2-16) &=3\\
  x^2-16 &=3^3\\
  x&= \pm\sqrt{43}
  \end{align*}
  \item \begin{align*} 
   2log_2(4-x)+4&=log_2(x+5)-1\\
   5&=log_2(x+5)-log_2(4-x)^2\\
   5&=log_2(\frac{x+5}{(4-x)^2})\\
   2^5&=\frac{x+5}{(4-x)^2}\\
   32(4-x) &= x+5\\
   32(16-8x+x^2)&=x+5\\
   512-256x+32x^2&=x+5\\
   32x^2-257x+507&=0\\
   x^2- \frac{257}{32}x + \frac{507}{32} &=0\\
   x_1&=\frac{1}{64}(257+\sqrt{1157}) \\
   x_2&=\frac{1}{64}(257-\sqrt{1157}) \\
   \end{align*} 
  
  \item
  \begin{align*} 
  log_5 x&=log_5 6-2log_5 3 \\
  0 &=log_5 6 -log_5 (3^2) - log_5 x \\
  0 &= log_5(\frac{6}{9x})\\
  1 &= \frac{2}{3x}\\
  x &= \frac{2}{3}\\
  \end{align*} 

  \end{enumerate}
  
  
  \item %SEITE 74 AUFGABE 3
  $N(t)=N_0 \cdot e^{kt}$ \newline
  f\"ur $\ N_0=100 \ und \ k=2 $\newline
  $N(t)=100\cdot e^{2t}$ \newline
  \begin{enumerate}
	\item 
	\begin{tabular}{rcccc}
	$t$ & & $ N(t)$ & & \\
	$5$ & & $100e^{10}$ & $\approx$ & $2,2 \cdot10^6$\\
	$10$ & & $100e^{20}$ & $\approx$ & $4,85 \cdot10^{10}$\\
	$20$ & & $100e^{40}$ & $\approx$ & $2,35 \cdot10^{19}$\\
	$50$ & & $100e^{100}$ & $\approx$ & $2,688 \cdot10^{45}$\\
	$100$ & & $100e^{200}$ & $\approx$ & $7,2 \cdot10^{88}$\\
	\end{tabular}
	
	
	\item
	A - Anzahl der Bakterien \newline
	\begin{align*}
	A &= 100 \cdot e^{2t}\\
	ln(\frac{A}{100}&=2t)\\
	t&= \frac{1}{2}(ln(A)- ln(100) \approx \frac{1}{2}(ln(A)-4,6)
	\end{align*}
	\begin{tabular}{rcccc}
	$A$ & & $ t$ & & \\
	$500$ & & $\frac{1}{2} ln(5)$ & $\approx$ & $0,8$\\
	$1000$ & & $\frac{1}{2} ln(10)$ & $\approx$ & $1,15$\\
	$5000$ & & $\frac{1}{2} ln(50)$ & $\approx$ & $1,96$\\
	$10000$ & & $\frac{1}{2} ln(100)$ & $\approx$ & $2,3$\\
	\end{tabular}
		\end{enumerate}
		Anmerkung: Rundungswerte sind nur zusätzliche Infos, ln-Ergebnisse können errechnet werden


	\item %SEITE 74 AUFGABE 4
	$N(t)=N_0 \cdot e^{- \lambda t}$ \newline
  $mit \ N_0=1000 \ , \ \lambda=2$ \newline
  $N(t)=1000\cdot e^{-2t}$ \newline
	
	\begin{enumerate}
	\item 
	\begin{tabular}{rcccc}
	$t$ & & $ t$ & & \\
	$1$ & & $1000 \cdot e^{-2}$ & $\approx$ & $135,3$\\
	$5$ & & $1000 \cdot e^{-10}$ & $\approx$ & $0,45$\\
	$10$ & & $1000 \cdot e^{-20}$ & $\approx$ & $2,06 \cdot 10^{-6}$\\
	$100$ & & $1000 \cdot e^{-200}$ & $\approx$ & $1,38 \cdot 10^{-84}$\\
	\end{tabular}
	\item 
	M = noch vorhandenes Material \newline
	$M=N(t)=1000 \cdot e^{-2t}$ \newline
  $t =-\frac{1}{2} ln(\frac{M}{1000})$\newline
  
  \begin{tabular}{rcccccc}
	$M$ & & & & $ t$ & & \\
	$\frac{1}{4}\cdot 1000$ &$=$ &250& & $-\frac{1}{2}ln(\frac{1}{4})$ & $\approx$ & $0,69$\\
	$\frac{1}{2}\cdot 1000$ &$=$ &500& & $-\frac{1}{2}ln(-\frac{1}{2})$ & $\approx$ & $0,35$\\
	$\frac{3}{4}\cdot 1000$ &$=$ &750& & $-\frac{1}{2}ln(-\frac{3}{4})$ & $\approx$ & $0,144$\\
	$\frac{7}{8}\cdot 1000$ &$=$ &875& & $-\frac{1}{2}ln(-\frac{7}{8})$ & $\approx$ & $0,0667$\\
	\end{tabular}
	
	
	
	
		\end{enumerate}
	
		
   \end{enumerate}
  
 \section{Kurvendiskusion}
  \begin{enumerate}
	\item %SEITE 78 AUFGABE 1
	\begin{enumerate}
	\item
 $f(x)=-x^{3}+3x-2 \newline
 f'(x)=-3x^2+3 \newline
 f''(x)=-6x \newline
 f'''(x)=-6 \newline$
 \par
 \bfseries Definitionsbereich: $\normalfont D(f)=\mathbb{R}$ \newline
 \bfseries Wertebereich: $\normalfont W(f)=\mathbb{R}$ \newline
 \bfseries Nullstellen: \normalfont \newline $ x_{1}=1 \newline x_{2}=-2  $\newline \newline
 \bfseries Extremstellen: \normalfont \newline 
 Minimum bei (-1;-4)  \newline
 Maximum bei (1;0) \newline \newline
 \bfseries Wendepunkte: \normalfont (0;-2)\newline 
 \bfseries Verhalten im Unendlichen: \normalfont \newline
 $\lim\limits_{x \rightarrow +\infty}{f(x)=-\infty}$ \newline
 $\lim\limits_{x \rightarrow -\infty}{f(x)=+\infty}$
 
 \par
 \item
  $f(x)=x^3-4x^2+5x-2 \newline
 f'(x)=3x^2-8x+5 \newline
 f''(x)=6x-8 \newline
 f'''(x)=6 \newline$
 
 \bfseries Definitionsbereich: \normalfont $D(f)=\mathbb{R}$ \newline
 \bfseries Wertebereich: \normalfont $W(f)=\mathbb{R}$ \newline
 \bfseries Nullstellen: \normalfont \newline $ x_{1}=1 \newline x_{2}=2  $\newline \newline
 \bfseries Extremstellen: \normalfont \newline 
 Minimum bei $(1\frac{2}{3};-\frac{4}{27})$  \newline
 Maximum bei (1;0) \newline \newline
  \bfseries Wendepunkte: \normalfont $(1\frac{1}{3};-\frac{}{27})$\newline 
 \bfseries Verhalten im Unendlichen: \normalfont \newline
 $\lim\limits_{x \rightarrow +\infty}{f(x)=+\infty}$ \newline
 $\lim\limits_{x \rightarrow -\infty}{f(x)=-\infty}$ \newline
 
  \end{enumerate} 
\item %SEITE 78 AUFGABE 2
  \begin{align*}
  g(x) &= \frac{3x^4-12x^3+9x^2+12x-12}{x^3-4x^2+5x-2}\\
  g(x) &= 3x + \frac{-6x^2+18x-12}{x^3-4x^2+5x-2} \ \text{Polynomdivision} \\
  \end{align*}
 \bfseries Definitionsl\"ucken: $x^3-4x^2+5x-2 \neq 0 \rightarrow x \neq 1; x\neq 2$  \newline
 \bfseries Definitionsbereich: \normalfont $D(f)=\mathbb{R} \wedge x\neq 1 \wedge x\neq 2$ \newline
 \bfseries Wertebereich: \normalfont $W(f)=\mathbb{R}$ \newline
 \bfseries Nullstellen: \normalfont \newline
 \begin{align*}
 3x^4-12x^3+9x^2+12x-12&=0\\
 x_1&=1\\
 x_2&=-1\\
 x_3&=2\\
 \end{align*}
 \bfseries Verhalten im Unendlichen: \normalfont \newline
 $\lim\limits_{x \rightarrow +\infty}{g(x)}=+\infty$ \newline
 $\lim\limits_{x \rightarrow -\infty}{g(x)}=-\infty$ \newline
 \bfseries Extremstellen: TODO \normalfont \newline
  \bfseries Wendepunkte: TODO\normalfont \newline
  

\item %SEITE 78 AUFGABE 3
  
  
     $f(x)=2\sin(x\pi) \newline
 f'(x)=2\pi \cos(x\pi)  \newline
 f''(x)=-2\pi^2\sin(x\pi)\newline
 f'''(x)=-2\pi^3\cos(x\pi) \newline$
 \bfseries Definitionsbereich: \normalfont $D(f)=\mathbb{R}$ \newline
 \bfseries Wertebereich: \normalfont $W(f)=\mathbb{R} \wedge -2\leq f(x) \leq 2$ \newline
 
 \bfseries Nullenstellen: \normalfont $1 \cdot k$ mit $k \in \mathbb{Z}$\newline
 \bfseries Extremstellen: \normalfont \newline
 Maximum: $(2k+0,5;2)$  mit $k \in \mathbb{Z}$ \newline
 Minimum: $(2k-0,5;-2)$ mit $k \in \mathbb{Z}$ \newline
  \bfseries Wendepunkte: \normalfont $(k;0)$  mit $ k \in \mathbb{Z}$\newline
  \newline
  Fl\"acheninhalt vom Dreieck ABC:\newline
  mit A(0;0) B(x;0) C(x;$2\sin (\pi x)$) \newline
  
  \begin{align*}
  A &= \frac{1}{2} \cdot g \cdot h \\
  A &= \frac{1}{2} \cdot x \cdot 2\sin (\pi x)  \\
  A &= x \cdot \sin (\pi x) \\
  A' &= \sin(\pi x)+x \pi \cos(\pi x)\\
  A'' &= 2 \pi \cos (\pi x) - \pi^2 x \sin(\pi x) \\
  A' &=0 \\
  x &\approx 0,64577 \\
  \end{align*}
  
  \end{enumerate} 