%================================================================================
\section{Übungsaufgaben}

%--------------------------------------------------------------------------------
\Aufgabe
Gegeben sind die komplexen Zahlen
\begin{align*}
z_1 &= -2i
& z_2 &= 3
& z_3 &= 1+2\imag
& z_4 &= 4-3\imag
\\
z_5 &= \e^{\pi/4}
& z_6 &= \e^{\imag\pi/4}
& z_7 &= 2\e^{-\frac{3\pi}4\imag}
& z_8 &= -\frac12\e^{\imag\cdot3\pi/2}
\end{align*}

\begin{enumerate}
\item{}Berechnen Sie jeweils ihren Betrag und ihr Hauptargument.

\item{}Rechnen Sie von der kartesischen in die Eulersche Form bzw.\ umgekehrt um.

\item{}Stellen Sie die Zahlen grafisch in der Gaußschen Zahlenebene dar.

\end{enumerate}

%--------------------------------------------------------------------------------
\Aufgabe
Berechnen Sie
\begin{enumerate}
\item$(1+2\imag)+(4-3\imag)$, $(2+4\imag)+3$, $(4+2\imag)-2\imag$

\item$(1+2\imag)\cdot(4-3\imag)$, $(3+2\imag)\cdot(3-2\imag)$, $(1+3\imag)\cdot(-1+3\imag)$

\item$\dfrac{1+2\imag}{4-3\imag}$, $\dfrac{3+2\imag}{3-2\imag}$, $\dfrac{1+3\imag}{-1+3\imag}$

\item $(1+\imag)^{4/2}$, $\left((1+\imag)^4\right)^{1/2}$

\item$\exp(1+2\imag)$, $\ln(1+2\imag)$

\end{enumerate}

%--------------------------------------------------------------------------------
\Aufgabe
Berechnen Sie
\begin{enumerate}
\item{}die Quadratwurzeln von $-\imag$ und von $\imag-1$,

\item{}die dritten Wurzeln von $8\e^{\frac{2\pi}3\cdot\imag}$,

\item{}die Nullstellen der Polynome $p_1(x)=x^5-x^4-2x^2-4x$ und $p_2(y)=y^4+3y^2+2$.

\end{enumerate}

%--------------------------------------------------------------------------------
\subsubsection*{Hausaufgabe}
\begin{enumerate}
\item{}Informieren Sie sich, welche Möglichkeiten zur Verarbeitung komplexer
Zahlen Ihre Lieblingsprogrammiersprache bzw.\ die Programmiersprache Java
bietet. Schreiben Sie gegebenenfalls ein Paket zur Arbeit mit komplexen
Zahlen.

\item{}Informieren Sie sich über Mandelbrot- und Julia-Mengen und das
"`Apfelmännchen"', und schreiben Sie ein Programm zur grafischen Darstellung
dieser Fraktale.

\end{enumerate}
