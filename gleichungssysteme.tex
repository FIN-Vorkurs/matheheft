\chapter{Lineare Gleichungssysteme}
	
	Autor: Marko Rak
	
	\section{Definition}
	
		Als \emph{lineares Gleichungssystem} bezeichnet man eine Menge von $m$ Gleichungen,
		die $n$ Unbekannte enthalten.
		Allgemein l\"asst sich solch ein Gleichungssystem immer in folgender Form darstellen:
		\[
			\begin{array} {ccccccccccc}
				a_{11} x_1 & + & a_{12} x_2 & + & a_{13} x_3 & + & \cdots & + & a_{1n} x_n & = & b_1\\
				a_{21} x_1 & + & a_{22} x_2 & + & a_{23} x_3 & + & \cdots & + & a_{2n} x_n & = & b_2\\
				a_{31} x_1 & + & a_{32} x_2 & + & a_{33} x_3 & + & \cdots & + & a_{3n} x_n & = & b_3\\
				\vdots & & & & & & \ddots & & & & \vdots\\
				a_{m1} x_1 & + & a_{m2} x_2 & + & a_{m3} x_3 & + & \cdots & + & a_{mn} x_n & = & b_m\\
			\end{array}
		\]
		
	\section{Lineare Abh\"angigkeit}
		
		Eine lineare Gleichung der obigen Form ist \emph{linear abh\"angig},
		wenn sie sich durch die anderen Gleichungen des Systems und der Multiplikation mit einer Konstanten $c_i$ darstellen l\"asst.
		\[
			\begin{array} {ccccccccccccc}
				& &a_{11} x_1 & + & a_{12} x_2 & + & \cdots & + & a_{1n} x_n & - & b_1 &  & \\
				 & = & (a_{21} x_1 & + & a_{22} x_2 & + & \cdots & + & a_{2n} x_n & - & b_2) & c_2 \\
				& + & (a_{31} x_1 & + & a_{32} x_2 & + & \cdots & + & a_{3n} x_n & - & b_3) & c_3\\
				 &\vdots &  & & & & \ddots  & & & & & \vdots\\
				& + & (a_{m1} x_1 & + & a_{m2} x_2 & + & \cdots & + & a_{mn} x_n & - & b_m) & c_n \\
			\end{array}
		\]
		
		Andernfalls ist sie \emph{linear unabh\"angig} von den anderen Gleichungen des Systems.
		
	\section{L\"osbarkeit}
	
		Ob ein lineares Gleichungssystem l\"osbar ist und wie viele L\"osungen es hat, ist unterschiedlich.
		Dabei tritt immer einer der folgenden F\"alle auf:
		
		\begin{enumerate}
			\item Das Gleichungssystem hat keine L\"osung.
			\item Das Gleichungssystem hat genau eine L\"osung.
			\item Das Gleichungssystem hat mehrere (meist unendlich viele) L\"osungen.
		\end{enumerate}
		
		\noindent Kriterien f\"ur die L\"osbarkeit und die Zuteilung eines linearen Gleichungssystems zu einem dieser F\"alle,
		w\"urde dem Vorlesungsinhalt vorgreifen und wird daher hier nicht ausf\"uhrlich erkl\"art.
		Allgemein l\"asst sich jedoch sagen:
		Hat ein lineares Gleichungssystem mehr Unbekannte als linear unabh\"angige Gleichungen,
		so hat es mehrere L\"osungen.
		
	\section{L\"osungsverfahren}
	
		Neben den bereits bekannten L\"osungsverfahren wie Gleich-, Einsetzungsverfahren usw., existieren noch weitere, systematische Verfahren.
		Dazu z\"ahlt u.A. das \emph{Gauss-Verfahren} (auch \emph{Gauss-Algorithmus} genannt),
		welches unter Verwendung einer vereinfachten Gleichungssystemdarstellung eine Diagonalform oder auch Dreiecksform erstellt.
		Diese beschleunigt das Finden von L\"osungen.
		
		\subsection{Vereinfachte Darstellung}
		
			Ein allgemeines lineares Gleichungssystem
			\[
				\begin{array} {ccccccccccc}
					a_{11} x_1 & + & a_{12} x_2 & + & a_{13} x_3 & + & \cdots & + & a_{1n} x_n & = & b_1\\
					a_{21} x_1 & + & a_{22} x_2 & + & a_{23} x_3 & + & \cdots & + & a_{2n} x_n & = & b_2\\
					a_{31} x_1 & + & a_{32} x_2 & + & a_{33} x_3 & + & \cdots & + & a_{3n} x_n & = & b_3\\
					\vdots & & & & & & \ddots & & & & \vdots\\
					a_{m1} x_1 & + & a_{m2} x_2 & + & a_{m3} x_3 & + & \cdots & + & a_{mn} x_n & = & b_m\\
				\end{array}
			\]
			l\"asst sich vereinfacht wie folgt darstellen:
			\[
				\begin{tabular} {ccccc|c}
					$x_1$ & $x_2$ & $x_3$ & $\cdots$ & $x_n$ &\\
					\hline
					$a_{11}$ & $a_{12}$ & $a_{13}$ & $\cdots$ & $a_{1n}$ & $b_1$\\
					$a_{21}$ & $a_{22}$ & $a_{23}$ & $\cdots$ & $a_{2n}$ & $b_2$\\
					$a_{31}$ & $a_{32}$ & $a_{33}$ & $\cdots$ & $a_{3n}$ & $b_3$\\
					$\vdots$ & & & $\ddots$ & & $\vdots$\\
					$a_{m1}$ & $a_{m2}$ & $a_{m3}$ & $\cdots$ & $a_{mn}$ & $b_m$\\
				\end{tabular}
			\]
			
			\noindent Nun werden die Unbekannten,
			da in allen Gleichungen der Systems gleich,
			nur noch im Tabellenkopf dargestellt.
			Tauchen in Gleichungen des Systems bestimmte Unbekannte nicht auf,
			werden sie in dieser Tabellendarstellung mit Faktor 0 aufgef\"uhrt.
			Das Gleichheitszeichen wird nun repr\"asentiert durch die Trennung vor der letzten Spalte.
			Auf die Additionsoperatoren wird gezielt verzichtet und die Subtraktion wird als Addition mit einem negativen   	Operanden betrachtet.
            Elementare Umformungen \"andern nichts an der L\"osung des linearen Gleichungssystems.
			Unter elementare Umformungen versteht man:

			\begin{enumerate}
				\item das Vertauschen von Spalten oder Zeilen
				\item die Multiplikation einer Zeile mit einer Konstanten
				\item die Addition eines Vielfachen einer Zeile zu einer anderen
			\end{enumerate}
						
		\subsection{Diagonalform}
		
			Die \emph{Diagonalform} des obigen allgemeinen linearen Gleichungssystems sieht wie folgt aus:
			\[
				\begin{tabular} {ccccccc|c}
					$x_1$ & $x_2$ & $x_3$ & $x_4$ & $\cdots$ & $x_{n-1}$ & $x_n$ &\\
					\hline
					$a_{11}$ & $a_{12}$ & $a_{13}$ & $a_{14}$ & $\cdots$ & $a_{1(n-1)}$ & $a_{1n}$ & $b_{1}$\\
					0 & $a_{22}^*$ & $a_{23}^*$ & $a_{24}^*$ & $\cdots$ & $a_{2(n-1)}^*$ & $a_{2n}^*$ & $b_{2}^*$\\
					0 & 0 & $a_{33}^*$ & $a_{34}^*$ & $\cdots$ & $a_{3(n-1)}^*$ & $a_{3n}^*$ & $b_{3}^*$\\
					0 & 0 & 0 & $a_{44}^*$ & $\cdots$ & $a_{4(n-1)}^*$ & $a_{3n}^*$ & $b_{3}^*$\\
					$\vdots$ & & & & $\ddots$ & & & $\vdots$\\
					0 & 0 & 0 & 0 & $\cdots$ & $a_{(m-1)(n-1)}^*$ & $a_{(m-1)n}^*$ & $b_{m-1}^*$\\
					0 & 0 & 0 & 0 & $\cdots$ & 0 & $a_{mn}^*$ & $b_{m}^*$\\
				\end{tabular}
			\]
			
			\noindent Mittels dieses Schemas lassen sich die L\"osungen des linearen Gleichungssystem relativ leicht erschlie\ss en.
			Man beginnt von unten und arbeitet sich zeilenweise aufw\"arts.
			Dabei kann mit jeder neuen Zeile eine weitere Unbekannte bestimmt werden.
			
			\noindent Aus der letzten Zeile
			\[
				a_{mn}^* x_n = b_{m}^*
			\]
			ergibt sich
			\[
				x_n = \frac {b_m^*} {a_{mn}^*}.
			\]
			Nun wird $x_n$ in die vorletzte Zeile
			\[
				a_{(m-1)(n-1)}^* x_{n-1} + a_{(m-1)n}^* x_n = b_{m-1}^*
			\]
			eingesetzt und nach $x_{n-1}$ umgestellt, was
            \[
				x_{n-1} = \frac {b_{m-1}^* -  \frac {a_{(m-1)n}^*} {a_{mn}^*} b_m^*} {a_{(m-1)(n-1)}^*}
			\]
			
			\noindent ergibt.
			Dies wird zeilenweise aufsteigend bis zur ersten Gleichung fortgesetzt,
			sodass gegebenenfalls alle Unbekannten ermittelt werden k\"onnen.
			
		\subsection{Gauss-Algorithmus}
		
			Der bereits angesprochende \emph{Gauss-Algorithmus} dient der Herstellung der Diagonalform aus einem beliebigen linearen Gleichungssystem.
			Dazu wird die vereinfachte Darstellung genutzt und mittels elementarer Umformungen schrittweise die Dreiecksform erstellt.
			Wir w\"ahlen in jedem Schritt eine Gleichung
			und addieren ein Vielfaches dieser zu jeder anderen Gleichung,
			um eine Spalte mit m\"oglichst vielen Nullen zu erzeugen.
			
			
			\noindent Die Ausgangssituation stellt sich wie folgt dar:
			
			\[
				\begin{tabular} {ccccc|c}
					$x_1$ & $x_2$ & $x_3$ & $\cdots$ & $x_n$ &\\
					\hline
					$a_{11}$ & $a_{12}$ & $a_{13}$ & $\cdots$ & $a_{1n}$ & $b_1$\\
					$a_{21}$ & $a_{22}$ & $a_{23}$ & $\cdots$ & $a_{2n}$ & $b_2$\\
					$a_{31}$ & $a_{32}$ & $a_{33}$ & $\cdots$ & $a_{3n}$ & $b_3$\\
					$\vdots$ & & & $\ddots$ & & $\vdots$\\
					$a_{m1}$ & $a_{m2}$ & $a_{m3}$ & $\cdots$ & $a_{mn}$ & $b_m$\\
				\end{tabular}
			\]
			
			\noindent Wir w\"ahlen die erste Gleichung aus und addieren ein Vielfaches davon zu den anderen Gleichungen,
			um in der ersten Spalte Nullen zu erzeugen.
			
			\[
				\begin{array} {ccccc|ccccc}
					x_1 & x_2 & x_3 & \cdots & x_n & & & & & \\
					\hline
					a_{11} & a_{12} & a_{13} & \cdots & a_{1n} & b_1 & \cdot(-\frac {a_{21}} {a_{11}}) & \cdot(-\frac {a_{31}} {a_{11}}) & \cdots & \cdot(-\frac {a_{m1}} {a_{11}})\\
					a_{21} & a_{22} & a_{23} & \cdots & a_{2n} & b_2 & \hookleftarrow & & &  \\
					a_{31} & a_{32} & a_{33} & \cdots & a_{3n} & b_3 & & \hookleftarrow & &  \\
					\vdots & & & \ddots & & \vdots & & & \ddots & \\
					a_{m1} & a_{m2} & a_{m3} & \cdots & a_{mn} & b_m & & & & \hookleftarrow  \\
				\end{array}
			\]
			
			\noindent Somit ergibt sich nach dem ersten Schritt diese Tabelle:
			
			\[
				\begin{tabular} {ccccc|c}
					$x_1$ & $x_2$ & $x_3$ & $\cdots$ & $x_n$ &\\
					\hline
					$a_{11}$ & $a_{12}$ & $a_{13}$ & $\cdots$ & $a_{1n}$ & $b_1$\\
					0 & $a_{22}'$ & $a_{23}'$ & $\cdots$ & $a_{2n}'$ & $b_2'$\\
					0 & $a_{32}'$ & $a_{33}'$ & $\cdots$ & $a_{3n}'$ & $b_3'$\\
					$\vdots$ & & & $\ddots$ & & $\vdots$\\
					0 & $a_{m2}'$ & $a_{m3}'$ & $\cdots$ & $a_{mn}'$ & $b_m'$\\
				\end{tabular}
			\]
			
			\noindent Nun w\"ahlen wir die zweite Gleichung und addieren ein Vielfaches davon zu jeder folgenden Gleichung,
			um in der zweiten Spalte auch Nullen zu erzeugen.
			
			\[
				\begin{tabular} {ccccc|cccc}
					$x_1$ & $x_2$ & $x_3$ & $\cdots$ & $x_n$ & & & & \\
					\hline
					$a_{11}$ & $a_{12}$ & $a_{13}$ & $\cdots$ & $a_{1n}$ & $b_1$ & & \\
					0 & $a_{22}'$ & $a_{23}'$ & $\cdots$ & $a_{2n}'$ & $b_2'$ & $\cdot (-\frac {a_{32}'}{a_{22}'})$ & $\cdots$ & $\cdot (-\frac {a_{m2}'}{a_{22}'})$\\
					0 & $a_{32}'$ & $a_{33}'$ & $\cdots$ & $a_{3n}'$ & $b_3'$ & $\hookleftarrow$ & & \\
					$\vdots$ & & & $\ddots$ & & $\vdots$ & & $\ddots$ & \\
					0 & $a_{m2}'$ & $a_{m3}'$ & $\cdots$ & $a_{mn}'$ & $b_m'$ & & & $\hookleftarrow$  \\
				\end{tabular}
			\]
			
			\noindent Was uns nach dem zweiten Schritt zu der folgenden Tabelle bringt:
			
			\[
				\begin{tabular} {ccccc|c}
					$x_1$ & $x_2$ & $x_3$ & $\cdots$ & $x_n$ &\\
					\hline
					$a_{11}$ & $a_{12}$ & $a_{13}$ & $\cdots$ & $a_{1n}$ & $b_1$\\
					0 & $a_{22}'$ & $a_{23}'$ & $\cdots$ & $a_{2n}'$ & $b_2'$\\
					0 & 0 & $a_{33}''$ & $\cdots$ & $a_{3n}''$ & $b_3''$\\
					$\vdots$ & & & $\ddots$ & & $\vdots$\\
					0 & 0 & $a_{m3}''$ & $\cdots$ & $a_{mn}''$ & $b_m''$\\
				\end{tabular}
			\]
			
			\noindent Dieser Ablauf wird wiederholt, bis die gew\"unschte Diagonalform entstanden ist
			und sich das erzeugte Schema wie oben beschrieben aufl\"osen l\"asst.
			
			\[
				\begin{tabular} {cccccc|c}
					$x_1$ & $x_2$ & $x_3$ & $\cdots$ & $x_{n-1}$ & $x_n$ &\\
					\hline
					$a_{11}$ & $a_{12}$ & $a_{13}$ & $\cdots$ & $a_{1(n-1)}$ & $a_{1n}$ & $b_{1}$\\
					0 & $a_{22}^*$ & $a_{23}^*$ & $\cdots$ & $a_{2(n-1)}^*$ & $a_{2n}^*$ & $b_{2}^*$\\
					0 & 0 & $a_{33}^*$ & $\cdots$ & $a_{3(n-1)}^*$ & $a_{3n}^*$ & $b_{3}^*$\\
					$\vdots$ & & & $\ddots$ & & & $\vdots$\\
					0 & 0 & 0 & $\cdots$ & 0 & $a_{mn}^*$ & $b_{m}^*$\\
				\end{tabular}
			\]
				
	\section{Beispiele}
	
		F\"ur alle Beispiele gilt $ x_i \in \mathbb{R} $ 
		
		\begin{enumerate}
				\item
				
					Die Ausgangssituation stellt sich wie folgt dar:
					\[
						\begin{array} {ccccccc}
							3 x_1 & - & 1 x_2 & + & 2 x_3 & = & 1\\
							7 x_1 & - & 4 x_2 & - & 1 x_3 & = & -2\\
							- x_1 & - & 3 x_2 & - & 12 x_3 & = & -5\\
						\end{array}
					\]
					
					und l\"asst sich vereinfacht darstellen:
					\[
						\begin{tabular} {ccc|c}
							$x_1$ & $x_2$ & $x_3$ &\\
							\hline
							3 & -1 & 2 & 1\\
							7 & -4 & -1 & -2\\
							-1 & -3 & -12 & -5\\
						\end{tabular}
					\]
					
					Jetzt wird schrittweise die Diagonalform erzeugt.
					Um Zeit und Platz zu sparen, lassen sich alle Schritte in einer Tabelle durchf\"uhren.
					\[
						\begin{tabular} {ccc|ccc}
							$x_1$ & $x_2$ & $x_3$ & & &\\
							\hline
							3 & -1 & 2 & 1 & $\cdot (-\frac{7} {3})$ & $\cdot (\frac{1} {3})$\\
							7 & -4 & -1 & -2 & $\hookleftarrow$ &\\
							-1 & -3 & -12 & -5 & & $\hookleftarrow$\\
							\hline
							3 & -1 & 2 & 1 & &\\
							0 & $-\frac{5} {3}$ & $-\frac{17} {3}$ & $-\frac{13} {3}$ & $\cdot (-2)$ &\\
							0 & $-\frac{10} {3}$ & $-\frac{34} {3}$ & $-\frac{14} {3}$ & $\hookleftarrow$ &\\
							\hline
							3 & -1 & 2 & 1 & &\\
							0 & $-\frac{5} {3}$ & $-\frac{17} {3}$ & $-\frac{13} {3}$ & &\\
							0 & 0 & 0 & 4 & &\\
						\end{tabular}
					\]
					
					Nach Herstellung der Diagonalform l\"asst sich das Ergebnis wie oben beschrieben leicht erschlie\ss en.
					In diesem Beispiel entsteht ein Widerspruch in der letzten Gleichung
					\[
						0 x_1 + 0 x_2 + 0 x_3 = 4.
					\]
					
					Somit hat das lineare Gleichungssystem keine L\"osung.
					
				\item
				
					Eine weitere, diesmal gleich vereinfachte, Aufgabenstellung.
					\[
						\begin{tabular} {ccc|c}
							$x_1$ & $x_2$ & $x_3$ &\\
							\hline
							2 & -5 & 3 & 3\\
							4 & -12 & 8 & 4\\
							3 & 1 & -2 & 9\\
						\end{tabular}
					\]
					
					Die schrittweise Umformung:
					\[
						\begin{tabular} {ccc|ccc}
							$x_1$ & $x_2$ & $x_3$ & & &\\
							\hline
							2 & -5 & 3 & 3 & $\cdot (-2)$ & $\cdot (-\frac{3} {2})$\\
							4 & -12 & 8 & 4 & $\hookleftarrow$ &\\
							3 & 1 & -2 & 9 & & $\hookleftarrow$\\
							\hline
							2 & -5 & 3 & 3 & &\\
							0 & -2 & 2 & -2 & $\cdot (\frac{17} {4})$ &\\
							0 & $\frac{17} {2}$ & $-\frac{13} {2}$ & $\frac{9} {2}$ & $\hookleftarrow$ &\\
							\hline
							2 & -5 & 3 & 3 & &\\
							0 & -2 & 2 & -2 & &\\
							0 & 0 & 2 & -4 & &\\
						\end{tabular}
					\]
					
					Es ergibt sich also nacheinander aus den letzten drei Gleichungen.
					\[
						\begin{array} {ccc}
							x_3 & = & -2\\
							x_2 & = & -1\\
							x_1 & = & 2\\
						\end{array}
					\]
					
					Somit hat das lineare Gleichungssystem genau eine L\"osung.

				\item
				
					Ein letztes Beispiel in aller K\"urze.
					\[
						\begin{tabular} {ccc|ccc}
							$x_1$ & $x_2$ & $x_3$ & & &\\
							\hline
							1 & -2 & 3 & 4 & $\cdot (-3)$ & $\cdot (-2)$\\
							3 & 1 & -5 & 5 & $\hookleftarrow$ &\\
							2 & -3 & 4 & 7 & & $\hookleftarrow$\\
							\hline
							1 & -2 & 3 & 4 & &\\
							0 & 7 & -14 & -7 & $\cdot (-\frac{1} {7})$ &\\
							0 & 1 & -2 & -1 & $\hookleftarrow$ &\\
							\hline
							1 & -2 & 3 & 4 & &\\
							0 & 7 & -14 & -7 & &\\
							0 & 0 & 0 & 0 & &\\
						\end{tabular}
					\]
					
					Es ist eine Nullzeile entstanden,
					welche auftritt, wenn zwei Gleichungen linear abh\"angig sind.
					Folglich hat das lineare Gleichungssystem nur noch 2 (linear unabh\"angige) Gleichungen und 3 Unbekannte.
					Es l\"asst sich eine Variable frei w\"ahlen, was zu unendlich vielen L\"osungen f\"ur dieses lineare Gleichungssystems f\"uhrt.
					Wir setzen also
					\[
						x_3 = t, t \in \mathbb{R}
					\]
					
					und l\"osen nun die anderen Unbekannten in Abh\"angigkeit von $t$ auf.
					\[
						\begin{array} {ccccc}
							x_2 & = & -1 & + & 2t\\
							x_1 & = & 2 & + & t\\
						\end{array}
					\]
				
		\end{enumerate}
		
		
	\section{Literatur}
	
 \begin{description}
     \item[Grundstudium.info]  \texttt{http://www.grundstudium.info/\allowbreak linearealgebra/\\\allowbreak lineare\_\allowbreak algebra\_grundlagennode9.php}
     \item[Mathematik.de] \texttt{http://www.mathematik.de/mde/fragenantworten/\\erstehilfe/linearegleichungssysteme/\\linearegleichungssysteme.html}
 \end{description}
   
		
		

		
  %\href{http://www.mathematik.de/mde/fragenantworten/erstehilfe/linearegleichungssysteme/linearegleichungssysteme.html}{http://www.mathematik.de/mde/fragenantworten/\\erstehilfe/linearegleichungssysteme/linearegleichungssysteme.html}
	
	\newpage
	\section{Aufgaben}
	
		F\"ur alle Aufgaben gilt $ x_i \in \mathbb{R} $ und $ a, b \in \mathbb{R} $ sind fest.
		
		\subsection{Einfache Gleichungssysteme}
			Bestimmen Sie die L\"osungen folgender Gleichungssysteme.

			\begin{enumerate}\abovedisplayskip-1em
				\item 
				
					\[
						\begin{array} {ccccccc}
							x_1 & + & 5 x_2 & + & 2 x_3 & = & 3 \\
							2 x_1 & - & 2 x_2 & + & 4 x_3 & = & 5\\
							x_1 & + & x_2 & + & 2 x_3 & = & 1\\
						\end{array}
					\]
				
				\item
				
					\[
						\begin{array} {ccccccc}
							7 x_1 & + & 8 x_2 & + & 5 x_3 & = & 3 \\
							3 x_1 & - & 3 x_2 & + & 2 x_3 & = & 1\\
							18 x_1 & + & 21 x_2 & + & 13 x_3 & = & 8\\
						\end{array}
					\]
				
				\item
					
					\[
						\begin{array} {ccccccccc}
							x_1 & + & x_2 & + & 3 x_3 & + & 4 x_4 & = & -3 \\
							2 x_1 & + & 3 x_2 & + & 11 x_3 & + & 5 x_4 & = & 2\\
							2 x_1 & + & x_2 & + & 3 x_3 & + & 2 x_4 & = & -3\\
							x_1 & + & x_2 & + & 5 x_3 & + & 2 x_4 & = & 1\\
						\end{array}
					\]	
						
				\item
					
					\[
						\begin{array} {ccccccc}
							x_1 & + & 2 x_2 & + & 3 x_3 & = & -4 \\
							5 x_1 & - & x_2 & + & x_3 & = & 0\\
							7 x_1 & + & 3 x_2 & + & 7 x_3 & = & -8\\
							2 x_1 & + & 3 x_2 & - & x_3 & = & 11\\
						\end{array}
					\]
						
				\item
					
					\[
						\begin{array} {cccccccccccc}
							- x_1 & + & x_2 & + & x_3 & & & - & x_5 & = & 0 \\
							x_1 & - & x_2 & - & 3 x_3 & + & 2 x_4 & - & x_5 & = & 2\\
							& & 3 x_2 & - & x_3 & - & 5 x_4 & - & 7 x_5 & = & 9\\
							3 x_1 & - & 3 x_2 & - & 5 x_3 & + & 2 x_4 & + & 5 x_5 & = & 2\\
						\end{array}
					\]	
	
				\item
					
					\[
						\begin{array} {ccccccccc}
							x_1 & - & 2 x_2 & - & 3 x_3 & & & = & -7 \\
							2 x_1 & - & x_2 & + & 2 x_3 & + & 7 x_4 & = & -3\\
							-2 x_1 & + & x_2 & + & 3 x_3 & + & 3 x_4 & = & 8\\
							x_1 & + & 4 x_2 & + & 5 x_3 & - & 2 x_4 & = & 7\\
						\end{array}
						\]
						
				\item
					
					\[
						\begin{array} {ccccccc}
							x_1 & - & x_2 & + & x_3 & = & 4 \\
							x_1 & + & 2 x_2 & + & x_3 & = & 13\\
							4 x_1 & + & 5 x_2 & + & 4 x_3 & = & 43\\
							2 x_1 & + & 4 x_2 & + & 2 x_3 & = & 26\\
						\end{array}
					\]
					

			\end{enumerate}
			
		\subsection{Parametrisierte Gleichungssysteme}
				
			Bestimmen Sie die L\"osungen folgender Gleichungssysteme in Abh\"angigkeit von $a$ und $b$.

			\begin{enumerate}\abovedisplayskip-1em
				\item
				
					\[
						\begin{array} {ccccccc}
							2 x_1 & - & x_2 & + & 4 x_3 & = & 0 \\
							x_1 & + & 3 x_2 & - & x_3 & = & 0\\
							7 x_1 & + & 7 x_2 & + & (4-a) x_3 & = & 0\\
						\end{array}
					\]
						
				\item
					
					\[
						\begin{array} {ccccccc}
							x_1 & + & x_2 & + & x_3 & = & 0 \\
							x_1 & + & a x_2 & + & x_3 & = & 4\\
							a x_1 & + & 3 x_2 & + & a x_3 & = & -2\\
						\end{array}
					\]
					
				\item
					
					\[
						\begin{array} {ccccccc}
							x_1 & - & 2 x_2 & + & 3 x_3 & = & 4 \\
							2 x_1 & + & x_2 & + & x_3 & = & -2\\
							x_1 & + & a x_2 & + & 2 x_3 & = & b\\
						\end{array}
					\]
					
			\end{enumerate}
