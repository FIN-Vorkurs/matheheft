\chapter{Funktionen}
\label{chap:funktionen}

Autor: Gerhard Gossen

\mbox{}\par

\noindent Eine Funktion $f$ ist eine Abbildung, die einem Wert aus dem \emph{Definitionsbereich $D(f)$} genau einen Wert aus dem \emph{Wertebereich $W(f)$} zuordnet. Die übliche Darstellung ist
$f : X \to Y$ (sprich: $f$ ist eine Abbildung von $X$ nach $Y$), wobei $X$ die Definitionsmenge ($D(f) \subseteq X$) und $Y$ die Zielmenge ist ($W(f) \subseteq Y$). Definitions- und Zielmenge sind oft $\R$ (die reellen Zahlen).

Verbreitete Funktionen sind z.B. Geraden ($f(x) = m\cdot x +n$), Polynome ($f(x) = a_n x^n + a_{n-1} x^{n-1}+\dots + a_1x+a_0$), die trigonometrischen Funktionen ($\sin x$, $\cos x$, $\tan x$, siehe Abschnitt~\ref{sec:trigonometrie}) oder Exponentialfunktionen ($a^x$, siehe Abschnitt~\ref{sec:exponential}). Abbildung~\ref{fig:funktionen} zeigt die Graphen einiger Funktionen.

\begin{figure}[bth]
\begin{center}
    \includegraphics[width=.5\textwidth]{funktionen}
\end{center}
\caption{Bekannte Funktionen}
\label{fig:funktionen}
\end{figure} 

Alle Funktionen, die wir im Vorkurs behandeln, sind Funktionen mit \emph{einer Ver"-änderlichen}, also Funktionen, die von einer einzigen Variable (meist $x$) ab"-hän"-gen. % In der Vorlesung werden Funktionen mit \emph{mehreren Veränderlichen} behandelt.

Die Eigenschaften einer Funktion kann man über eine \emph{Kurvendiskussion} (siehe Abschnitt~\ref{sec:kurvendiskussion}) herausbekommen. Zuerst werden wir aber zwei wichtige Funktionsfamilien vorstellen: die trigonometrischen Funktionen (Winkelfunktionen, Abschnitt~\ref{sec:trigonometrie}) und die Exponentialfunktionen (Abschnitt~\ref{sec:exponential}).