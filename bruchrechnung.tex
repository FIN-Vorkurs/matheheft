\chapter{Basismathematik}
\addtocontents{toc}{\setcounter{tocdepth}{1}}

\section{Bruchrechnung}
\label{Bruchrechnung}
Autor: Katja Matthes
%%%%%%%%%%%%%%%%%%%%%%%%%%%%%%%%%%%%%%%%%%%%%%%%%%%%%%%%%%%%%%%%%%%%
%%%Definition
%%%%%%%%%%%%%%%%%%%%%%%%%%%%%%%%%%%%%%%%%%%%%%%%%%%%%%%%%%%%%%%%%%
\subsection{Definition}
Ein Bruch ist die Darstellung einer rationalen Zahl als Quotient.

\qquad Bruch: $\frac{Z}{N}$ \qquad mit $Z \in \mathbb{Z}$ und $N \in \mathbb{Z}\setminus\{0\} $
	
\qquad Z\ \ldots\ Zähler \qquad N\ \ldots\ Nenner
	
Zwei Brüche $\frac{a}{b}$ und $\frac{c}{d}$ hei\ss en gleichnamig, wenn sie den gleichen Nenner haben: $b = d$.

%%%%%%%%%%%%%%%%%%%%%%%%%%%%%%%%%%%%%%%%%%%%%%%%%%%%%%%%%%%%%%%%%%%%
%%%K�rzen und Erweitern
%%%%%%%%%%%%%%%%%%%%%%%%%%%%%%%%%%%%%%%%%%%%%%%%%%%%%%%%%%%%%%%%%%
\subsection{Kürzen und Erweitern}
Ein Bruch wird gekürzt, indem sowohl Nenner als auch Zähler durch die gleiche Zahl dividiert werden.
\[ \frac{a \cdot c}{b \cdot c} \stackrel{: c}{=} \frac{a}{b}\]
Ein Bruch wird erweitert, indem sowohl Nenner wie Zähler mit dem gleichen Faktor multipliziert werden.
\[ \frac{a}{b} \stackrel{\cdot c}{=}  \frac{a \cdot c}{b \cdot c} \]
	
%%%%%%%%%%%%%%%%%%%%%%%%%%%%%%%%%%%%%%%%%%%%%%%%%%%%%%%%%%%%%%%%%%%%
%%%spezielle Rechenregeln
%%%%%%%%%%%%%%%%%%%%%%%%%%%%%%%%%%%%%%%%%%%%%%%%%%%%%%%%%%%%%%%%%%
\subsection{Spezielle Rechenregeln}
\subsubsection{Addition von gleichnamigen Brüchen}
Zwei gleichnamige Brüche werden addiert, indem ihre Zähler addiert werden und der Nenner übernommen wird.
	\[\frac{a}{b} + \frac{c}{b} = \frac{a+c}{b}\]
	
\subsubsection{Subtraktion von gleichnamigen Brüchen}
Zwei gleichnamige Brüche werden subtrahiert, indem ihre Zähler subtrahiert werden und der Nenner beibehalten wird.
	\[\frac{a}{b} - \frac{c}{b} = \frac{a-c}{b}\]
	
\subsubsection{Multiplikation mit einem Faktor}
Ein Bruch wird mit einem Faktor $n$ multipliziert, indem der Zähler mit diesem Faktor multipliziert wird, während der Nenner übernommen wird.
	\[\frac{a}{b} \cdot n = \frac{a \cdot n}{b}\]
	
\subsubsection{Division durch eine Zahl}
Ein Bruch wird durch eine Zahl $n \neq 0$ dividiert, indem der Nenner mit dieser Zahl multipliziert wird und der Zähler beibehalten wird.
	\[\frac{a}{b} : n = \frac{a}{b \cdot n}\]

%%%%%%%%%%%%%%%%%%%%%%%%%%%%%%%%%%%%%%%%%%%%%%%%%%%%%%%%%%%%%%%%%%%%
%%%allgemeine Rechenregeln
%%%%%%%%%%%%%%%%%%%%%%%%%%%%%%%%%%%%%%%%%%%%%%%%%%%%%%%%%%%%%%%%%%	
\subsection{Allgemeine Rechenregeln}
\subsubsection{Addition}
Zwei Brüche werden addiert, indem sie zunächst gleichnamig gemacht werden und dann die Zähler addiert werden.
	\[\frac{a}{b} + \frac{c}{d} = \frac{a \cdot d}{b \cdot d} + \frac{b \cdot c}{b \cdot d} = \frac{a \cdot d + b \cdot c}{b \cdot d} \]
	
\subsubsection{Subtraktion}
Zwei Brüche werden subtrahiert, indem sie zunächst gleichnamig gemacht werden und dann die Zähler subtrahiert werden.
	\[\frac{a}{b} - \frac{c}{d} = \frac{a \cdot d}{b \cdot d} - \frac{b \cdot c}{b \cdot d} = \frac{a \cdot d - b \cdot c}{b \cdot d} \] 
	
\subsubsection{Multiplikation}
Zwei Brüche werden mulipliziert, indem jeweils die Nenner und Zähler multipliziert werden.
	\[\frac{a}{b} \cdot \frac{c}{d} = \frac{a \cdot c}{b \cdot d}\]

\subsubsection{Division}
Ein Bruch wird durch einen anderen dividiert, indem er mit dessen Kehrwert multipliziert wird.
\[\frac{a}{b} : \frac{c}{d} = \frac{a}{b} \cdot \frac{d}{c} = \frac{a \cdot d}{b \cdot c}\] 



%%%%%%%%%%%%%%%%%%%%%%%%%%%%%%%%%%%%%%%%%%%%%%%%%%%%%%%%%%%%%%%%%%%%
%%%Aufgaben
%%%%%%%%%%%%%%%%%%%%%%%%%%%%%%%%%%%%%%%%%%%%%%%%%%%%%%%%%%%%%%%%%%
\newpage
\subsection{Aufgaben}
\subsubsection{Aufgabe 1}
Kürze soweit möglich.
\begin{enumerate}
\begin{multicols}{4}
	\item \quad $ \frac{20}{6} $
	\item \quad $ \frac{92}{4} $
	\item \quad $ \frac{360}{25} $
	\item \quad $ \frac{1716}{308} $
\end{multicols}
\end{enumerate}

\subsubsection{Aufgabe 2}
Berechne und kürze soweit wie möglich.
\begin{enumerate}
\begin{multicols}{2}
	\item \quad $ \frac{56}{65} \cdot 12 \cdot \frac{5}{7} \cdot \frac{13}{16} $
	\item \quad $ 1 : \left( \frac{2}{9} + \frac{1}{7} \right) $
	\item \quad $ \left( \frac{3}{5} - \frac{1}{4} \right) : \frac{3}{4} $
\end{multicols} 
\end{enumerate}

\subsubsection{Aufgabe 3}
Berechne.
\begin{enumerate}
\begin{multicols}{4}
	\item \quad $ \frac{\frac{8}{9}}{\frac{16}{27}} $
	\item \quad $ \frac{2\frac{1}{3}}{1\frac{1}{6}} $
	\item \quad $ \frac{5\frac{1}{2}}{\frac{11}{12}} $
	\item \quad $ \frac{\frac{99}{100}}{\frac{9}{10}} $
\end{multicols} 
\end{enumerate}

\subsubsection{Aufgabe 4}
Berechne.
\begin{enumerate}
	\item \quad $ \frac{5}{6} \cdot \frac{2}{3} - \frac{2}{9} + \frac{3}{4} \cdot 1\frac{7}{9} $
	\item \quad $ 3\frac{5}{12} - 2\frac{5}{6} + 1\frac{1}{3} : \frac{4}{9} - 2\frac{1}{6} \cdot \frac{1}{2} $
\end{enumerate}

\subsubsection{Aufgabe 5}
Berechne.
\begin{enumerate}
\begin{multicols}{2}
	\item \quad $ \left(\frac{2}{3} - \frac{1}{6}\right) \cdot \left(\frac{9}{11} - \frac{3}{7}\right) $
	\item \quad $	\left(\frac{1}{8} + \frac{7}{12}\right) : \left(5 - \frac{3}{4}\right) $
	\item \quad $ \frac{4}{7} \cdot \left(\left(1\frac{1}{2} - \frac{5}{9}\right) : 4\frac{1}{4}\right) $
	\item \quad $ \frac{4}{5} : \left[\left(\frac{5}{8} - \frac{1}{3}\right) \cdot 12\right] $
	\item \quad $ \frac{3}{4} \cdot \left(2\frac{1}{2} : 1\frac{1}{4}\right) $
\end{multicols} 
\end{enumerate}

\subsubsection{Aufgabe 6}
\vspace{-0.15cm}
Berechne.
\vspace{-0.3cm}
\begin{enumerate}
\begin{multicols}{2}
	\item \quad $ \frac{\frac{3}{8} \cdot \frac{2}{7}}{\frac{5}{14}} $
	\item \quad $ \frac{1\frac{3}{4} + \frac{5}{6}}{\frac{1}{4}} $
	\item \quad $ \frac{\frac{8}{9}}{3\frac{1}{3} + \frac{1}{6}} $
	\item \quad $ \frac{\left(\frac{3}{5} - \frac{5}{10}\right) : \frac{2}{5}}{\frac{1}{4} + \frac{1}{2}} $
\end{multicols}
\end{enumerate}

