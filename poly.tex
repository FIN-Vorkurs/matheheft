\polyset{style=C, div=:}
\newcommand{\grad}{\ensuremath{\text{grad}\,}}
\newcommand{\halb}{\tfrac{1}{2}}
\section{Polynomdivision}

	Autor: Gerhard Gossen
	
\noindent	\"Uberarbeitung: Marko Rak
	
	\subsection{Polynome}
	
		Ein Polynom ist ein Term der Form
		\[a_nx^n + a_{n-1}x^{n-1} + \dots + a_2x^2 + a_1x + a_0 \quad a_n \neq 0\]
		wobei die $a_i \in \mathbb{R}$, $n \in \mathbb{N}$ und $x$ variabel sind.
		
		Der Grad eines Polynoms ($\grad p(x)$) ist der höchste Exponent von
		$x$. Beispielsweise ist $\grad(3x^2+2x^5-25x) = 5$.
		
	\subsection{Verfahren}
		
		Gegeben sind zwei Polynome $p(x)$ und $q(x)$. Die Division
		$p(x) : q(x)$ ergibt zwei neue Polynome:
		\[p(x) : q(x) = s(x) + \frac{r(x)}{q(x)}.\]
		Dabei ist $r(x)$ der \glqq Rest\grqq{} der Division.
		
		Bei der Berechnung entfernt man die höchsten Terme nacheinander. Dazu sucht
		man einen Term $s_k = b_kx^k$, der mit dem ersten Term von $q$ multipliziert
		den ersten Term von $p$ ergibt. Diesen Term multipliziert man mit $q$ und
		subtrahiert ihn von $p$. Der entstehende Term $p'$ ist vom Grad kleiner als
		$p$. $s_k$ wird zum ersten Term von $s(x)$ (dem \glqq Ergebnispolynom\grqq).
		Dieses Verfahren führt man solange durch wie möglich, also solange
		$\grad p'(x) \geq \grad q(x)$.
		
		\subsection{Beispiel}
			Berechnet werden soll $(-3-3x^2+x+x^3):(1+x)$.
			
			Zuerst ordnen wir die Polynome nach Exponenten: $(x^3-3x^2+x-3):(x+1)$. Im
			ersten Schritt wird also $x^3$ entfernt, der erste Ergebnisterm ist damit
			$x^2$, da $x^2\cdot x = x^3$. Damit subtrahieren wir $x^2(x+1) = x^3+x^2$. 
			
			\polylongdiv[stage=4]{x^3-3x^2+x-3}{x+1}
			
			Jetzt müssen wir also nur noch $(-4x^2+x-3):(x+1)$ berechnen. Wir rechnen
			analog solange wie möglich weiter.
			
			\polylongdiv[stage=10]{x^3-3x^2+x-3}{x+1}
			
			Wir berechnen jetzt $-8:(x+1)$. Da $\grad(-8) < \grad(x+1)$, bricht die
			Polynomdivision hier ab. $-8$ ist der \glqq Rest\grqq{} $r(x)$ der Berechnung.
			
			\polylongdiv{x^3-3x^2+x-3}{x+1}
			
			Das Ergebnis von $(x^3-3x^2+x-3):(x+1)$ ist damit $x^2-4x+5+\frac{-8}{x+1}$.
			Als Probe multiplizieren wir das Ergebnis mit $(x+1)$.
			\begin{align*}\hspace{-11pt}
				(x^2-4x+5+\frac{-8}{x+1})(x+1) 
				&= x^2(x+1)-4x(x+1)+5(x+1)+\frac{-8}{x+1}(x+1)\\
				&= (x^3 + x^2) + (-4x^2 -4x) + (5x +5) + (-8)\\
				&= x^3 -3x^2 +x -3
			\end{align*}
			Dies ist unser ursprüngliches Polynom, wir haben also richtig gerechnet.
		
	\subsection{Weitere Beispiele}
	
		\polylongdiv{4x^5-x^4+2x^3+x^2-1}{x^2+1}
		
		\polylongdiv{x^4 + 2x^3 - 3x^2 - 8x - 4}{x^2 - 4}
		
	\subsection{Aufgaben}
	
		Berechne:
		\begin{enumerate}
			\item $(x^3+1):(x+1)$
			\item $(x^4-x+1):(x^2+x+1)$
			\item $(x^2-9):(x+3)$
			\item $(6x^3-5x^2-36x+35):(3x-7)$
			\item $(x^5-x^2-2x+1) : (x^4-x^3+2x^2-3x+1)$
			\item $(x^5-x^3+x^2+x-2) : (x^2-1)$
			\item $(3x^3+2x^2+4x+9) : (3x+5)$
			\item $(2x^5 + 8x^4 +x^3-x^2 + 12 x +3): (x^2+4x+1)$
			\item $(x^6-2x^5+9x^4-8x^3+15x^2):(x^2-x+5)$
			\item $(2x^7 - x^6 + 3x^5 - \tfrac{1}{2}x^4 + x^3) : (2x^3 - x^2 + 2x)$
			\item $(x^7-6x^5+x^4-11x^2-3x+1):(x^3+2)$
			\item $(3x^5 +6x^4 +\tfrac{11}{3}x^3 + 4x^2 + \tfrac{20}{3}x):(3x^4 +x^3+4x)$
			\item $(\tfrac{1}{6}x^4 + \tfrac{11}{36}x^3 - \tfrac{23}{18}x^2 -
			\tfrac{1}{3}x + \tfrac{2}{3}) : (\tfrac{1}{2}x^2 - \tfrac{4}{3}x +
			\tfrac{2}{3})$
			\item $(\tfrac{5}{4}x^4- \tfrac{1}{4}x^2  + \halb x^5 + \halb x^3 -\halb x)
			: (\halb x^2 +x)$
			\item $(\halb x^5 - \tfrac{3}{4}x^4 - \tfrac{1}{4}x^3 + \tfrac{3}{4}x^2 -
			\tfrac{15}{4}x + \tfrac{7}{4}) : (\halb x -\tfrac{1}{4})$
		\end{enumerate}
		
	\subsection{Literatur}
	
		Beispiele erstellt mit dem \LaTeX-Paket \glq polynom\grq{} von Donald
		Arseneau (\url{http://www.atscire.de/index.php?nav=products/polynom}).
		
		Website mit kommentierter Rechnung und Aufgaben zum Selberrechnen:
		\url{http://www.arndt-bruenner.de/mathe/scripts/polynomdivision.htm}.